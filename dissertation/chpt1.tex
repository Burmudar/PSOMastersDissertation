\chapter{Introduction}
\section{Introduction}
In the technology age, life is almost unfathomable without the mobile phone. It is hard to believe how the business world managed to function in the pre-mobile phone era\footnote{Pre 1980s where the mobile phone did not yet exist}. 

The invention of the mobile phone fulfilled the need to always be connected and be within reach of the modern world. This need can actually be attributed to feeling part of something and this something is deemed as being part of a network. This is similar to how mobile phones are able to provide connectivity at almost any location within a country.

Behind the connectivity of mobile phones lies an intricate layer of communication technology which has evolved since the invention of the normal landline phone. This technology is referred to at the top level as wireless communication, which includes inventions such as the radio. To enable communication at the level needed by mobile phones, wireless communication had to evolve and go a step further than normal radio communication. Hence the concept of cellular mobile networks was developed.

Without cellular networks mobile phones have no means of providing communication. The purpose of the cellular network is to provide the necessary functions to connect and facilitate communication between two entities (mobile phones) over a wireless network. 

Cellular networks achieve this level of communication through expensive equipment and ``smart'' algorithms. By ``smart'' algorithms, what is actually meant is algorithms that utilise artificial intelligence concepts to perform a certain function in an attempt to either automate a task within the network or improve a certain part of the network. Therefore, the correct term for these types of algorithms is actually artificial intelligence (AI) algorithms.

AI algorithms have a wide spread of functions that they can perform ranging from making informed intelligent decisions to optimising the operation of certain processes. In particular computers are indeed very apt at optimising certain procedures. This is because a computer is able to test and evaluate a huge number of different alterations and combinations of a certain procedure in a short amount of time, thereby allowing it to find the best combination out of those tested.

In this dissertation an algorithm is presented which concentrates on cellular phone networks. This particular algorithm falls into the latter part of the AI algorithms discussed, namely an AI optimisation algorithm. Thus the algorithm presented in this dissertation operates on a cellular phone network to optimise a certain part of the network.

In this section a introduction was presented to the dissertation. The main themes of the dissertation were outlined namely, cellular phone networks and optimisation algorithms. In the next section research plan and chapter breakdown is presented which is followed by this dissertation.
\section{Research Questions and Objective}
As previously discussed, an artificial intelligent algorithm is presented in this dissertation that generates plans for use by cellular networks. The purpose of creating such an algorithm is to determine the applicability of modern swarm-based algorithms with regard to finding better solutions to real-world problems that exist in domains such as cellular communication.

The research approach followed in this dissertation was first and foremost to understand the cellular network domain. More importantly, the focus is on how and what frequencies are used to facilitate communication as well as what affects the quality of the communicational link between two entities in a cellular network. 

The cellular domain is the problem domain, but to develop a solution in the problem domain a study needs to be conducted in the artificial intelligence domain, namely a study of optimisation algorithms. In order for a new optimisation algorithm to be developed, other algorithms that have achieved success in the respective optimisation problem domains in which they have been applied need to be investigated.

Therefore in an attempt to develop and apply a modern viable optimisation-based algorithm, a series of research questions has been identified which need to be answered. The research questions are as follows:
\begin{itemize}
\item \textbf{Question 1} --- What is cellular technology, how it got developed and what improvements have been made since its initial development?
\item \textbf{Question 2} --- What is the architecture behind a modern cellular network, how do the various hardware entities within a network communicate with each other and how is a communicational link established between two users of the network?
\item \textbf{Question 3} --- What exactly is the frequency problem and how does it affect modern wireless communication?
\item \textbf{Question 4} --- What variants of the frequency problem exist and which are most applicable to cellular networks?
\item \textbf{Question 5} --- What are the most popular optimisation algorithms and what characteristics make them unique?
\item \textbf{Question 6} --- With algorithms that achieved success in their respective optimisation problems, what particular technique is used by the algorithms that allowed them to achieve better performance?
\end{itemize}

The PSO algorithm might be short and elegant, but applying it to the FS-FAP required various new techniques in response to the following questions:
\begin{itemize}
\item How can a particle best be represented as a frequency plan?
\item How can one frequency plan move to another?
\item How can particles be prevented from using forbidden frequencies when they fly towards a particular plan?
\end{itemize}

The above questions were only the preliminary questions and were in fact a problem that had to be addressed for successful application of the PSO to the FAP. In chapter~\ref{chpt:psoapplicationFAP} a discussion is presented on how these problems were solved as well as how other problems were solved that were not anticipated.

Throughout the course of this dissertation the aim is to answer each of these identified research questions. In the next section a chapter breakdown of this dissertation is presented.

The aim of this dissertation is first and foremost to answer the identified research questions. Answering these questions will aid in reaching the research objective of this dissertation. The research objective is therefore to develop and determine the viability of an algorithm based on modern swarm optimisation techniques that will operate on the frequency assignment problem in an attempt to provide better frequency plans for use in cellular networks.

\section {Chapter Breakdown}
\subsection{Part I - Background}
The first part of this dissertation is concerned with the domain and problem the algorithm presented in this dissertation addresses, and finally to also understand the intricate details of how the algorithm operates.

Below is an outline of the chapters in the first part of this dissertation.
\subsubsection{Chapter 1}
This chapter provides an introduction to the dissertation as well as a broad overview of the topics in the dissertation will discuss.
\subsubsection{Chapter 2}
This chapter is concerned with providing information on how a modern cellular network functions. Within this chapter a brief history is presented on how cellular network technology was developed. The chapter also provides an overview of the architecture of a cellular network,and each part of the network's intended purpose and function to facilitate wireless communication is discussed.
\subsubsection{Chapter 3}
This chapter presents the problem that the dissertation addresses, namely the frequency assignment problem. The chapter provides a discussion on why the problem exists, the causes of the problem and what it means for a problem to be NP-Complete. Furthermore the variants of the problem and how they differ depending on the wireless domain that is being considered are alos discussed. Finally the chapter also provides a formal definition of the problem which is later utilised by the algorithm developed in this research.
\subsubsection{Chapter 4}
This chapter marks the beginning of a discussion on various optimisation algorithms in this dissertation. The algorithms presented in this chapter were chosen due to their widespread usage as well as success on NP-Complete problems. Each algorithm is discussed in depth providing an outline of the core features that make the algorithm unique as well as each core feature in detail. For each algorithm the chapter also presents an analysis on related work of the particular algorithm being applied to the frequency assignment problem. 
\subsubsection{Chapter 5}
This chapter is concerned with providing algorithms that are new in the research domain of optimisation algorithms. The algorithms presented in chapter 4 are fairly old and have been applied to a wide variety of problems. Algorithms in this chapter are relatively new in the optimisation domain and have not been applied to the same number of problems as the algorithms in chapter 4. In this chapter swarm algorithms are presented and the algorithms have the particular characteristic that they are based generally on processes observed in nature. Each algorithm is discussed in depth with its core characteristics outlined. Furthermore for each algorithm an analysis is given if the algorithm were to be applied to the frequency assignment problem. Finally it is formally stated what algorithm this dissertation is applied to the frequency assignment problem in this research.
\subsection{Part II - Implementation}
\subsubsection{Chapter 6}
This chapter provides a discussion of the algorithm developed to be applied to the frequency assignment problem. Within this chapter an outline is given of the process in developing a specialised particle swarm algorithm for the frequency algorithm. Each specialised technique developed is discussed in depth, along with an explanation of why the technique is needed as well as why it is used by the algorithm.
\subsubsection{Chapter 7}
This chapter is concerned with providing the results after applying the algorithm to a specialised set of benchmark problems for frequency assignment algorithms. The particular selected benchmark problems were discussed in chapter 2.
\subsubsection{Chapter 8}
This chapter concludes this dissertation. In this chapter it is determined whether the research goal was reached as well as whether any future work can be done to improve the presented algorithm.
