\chapter{Cellular Technology}
\section{Introduction}
In the information age we currently live in almost every device has some sort of wireless technology it uses to provide a specific service. Radios for audio entertainment; Television remotes to change 
channels; Cellular phones for communication; Wireless access points to create wireless LAN's\cite{Karen2004}. Wireless technology is now part of our everyday life.

The popularity and rapid adoption of wireless technology hasn't been kind to the management, planning and operation of wireless networks; It has actually worsen a problem known as the Frequency Assignment problem (FAP) which is present in all forms of wireless communication especially in GSM Cellular networks.

In this chapter we will start of by giving a brief history of the Frequency Assignment Problem. After the brief history overview we will give a description of the problem and explain some of the
underlying concepts needed to understand the Frequency Assignment Problem. We will then move on to discuss the different variants present in the current domain. Further more we will
then discuss the two underlying approaches used in solving the Frequency Assignment Problem (FAP).
\section{GSM Networks}
The General System for Mobile Communications (GSM) is a system for multi-service cellular communication which is capable of providing voice as well as data services. Most cellular networks in operation 
are GSM based. The primary service that GSM caters for is voice communication, but other data services such as Short Message Service (SMS), Multimedia Message Service (MMS) and Internet 
connectivity services such as GPRS are becoming more important\cite{Eisenblatter}.

GSM is one of the most widely used radio communications technologies, which is why we need to look at the history behind it in order for us to understand the domain of radio communication better.We will now present a brief history of the GSM network specification.
\subsection{A Brief History of GSM Networks}
In the early 1981 a group known as the Groupe Speciale Mobile (GSM) was established to develop a Europe wide radio communication system using the reserved 900 MHz band\footnote{In 1990 the United Kingdom requested that 1800MHz band be added to the scope of the GSM standard group. This variant of the GSM specification became known as the \emph{Digital Cellular System 1800} (DCS1800) \cite{GSM92}.}.

At the start of the GSM specification in the early 1980's it was initially thought that the system would be analogue based, but this soon changed with the \emph{Integrated Service Digital Network} (ISDN) specification nearing completetion. As such the GSM specification started following much of the same design principles and access protocols that ISDN exhibited. After the completion of the ISDN specification and the advantages it brought to the field, it became unofficially clear that GSM would be based on digital transmission and that speech would be represented by a digital stream of 16 kbits/s \cite{GSM92}.

Before the switch to digital transmission was finalized the GSM first wanted to evaluate the spectral efficiency of analogue and digital based transmission. Spectral efficiency plays an important part in wireless communication since the radio spectrum is a limited resource and whichever transmission technology is used, should maximise the utilization of the spectrum. Maximum utilisation is an important problem which we will discuss in detail in later sections of this chapter. The Spectral evaluation was conducted over a period of 3 years from 1984-1987. In 1987 a report was published about the 3 year evaluation and subsequently it became official that the GSM system would be digital based using \emph{Time Division Multiple Access} (TDMA) \cite{GSM92,GSMSysEngin}.

By the early 1990’s GSM became an evolving standard and the first GSM based network was demonstrated in 1991\footnote{Near the end of 1991 the GSM group was renamed to \emph{Speciale Mobile Group} (SMG) to eliminate confusion with the standard and the group.}. The following year a number of GSM networks were operating in Europe due to mobile terminals / equipment capable of operating on the networks becoming more widely available to the general public. In the same year an operator in Australia became the first non-European operator to implement a GSM based network\cite{Eisenblatter}.

The collective subscriber base of GSM networks surpassed the million subscriber mark in 1993. Due to this phenomenal growth in GSM network use, numerous extensions were made to the GSM specification. 
Some of the extensions that were made are the following\cite{Eisenblatter}:
\begin{itemize}
\item Half rate speech telephony
\item Improved SMS
\item Line Identification
\item Call waiting
\item Call holding
\end{itemize}
The specification with these extensions defined is known as the GSM Phase 2. As the world shifted towards more digital and data intensive services it became difficult to deliver these services over GSM 
networks. This difficulty was due to the restriction that data could only be transmitted at 9.6 Kbps. A move to eliminate this restriction was made with the specification of GSM Phase 2+. 

The new specification defined new technologies such as General Packet Radio System (GPRS) and \emph{Enchanced Data rate for GSM Evolution} (EDGE) which were designed with the primary goal of making more bandwidth available for data transmission.
These new technologies have an inherent requirement that there be a higher signal to noise ratio present at transceivers. This requirement has an impact that effects radio interfaces and more 
importantly Frequency planning \cite{Eisenblatter}. 

The actual signal to noise ratio at a receiver is dependent on a number of factors that include \cite{Karen2004}:
\begin{itemize}
\item Frequency used at the transceiver
\item Strength of the signal
\item Weather conditions
\item Shape of the surrounding environment
\item Direction of the transmission
\end{itemize}
Even taking these factors into account the calculation of the signal to noise ratio at a transceiver is not trivial. For a more in depth discussion on the calculation the reader is directed at the survey by \cite{Karen2004}.

As the GSM standard matured as a cellular technology, industry experts already began specification of the next generation of cellular networks which would in time, replace the GSM cellular system. The specification of a new standard is considered to be a natural evolution of the technology. Each standard is designed with specific use cases in mind as to what its users might want to do as well as what is possible with the technology at the designers disposal. As time goes by, the technology improves and users habits and needs change, thus the standard must be improved upon to serve these new needs and incorporate new technology.

The \emph{Universal Mobile Telecommunications System} (UMTS) can be considered the 3rd generation (3G) of cellular networks. UMTS was designed from the beginning to operate in parallel with the legacy GSM system. This decision was made to make the deployment of the system as hassle free as possible for the network operators. The first standard of the UMTS was issued in the beginning of 2000 and subsequently most modern networks are based on it or are migrating their networks to it.

UMTS is a large improvement of the GSM in two areas namely Data Transmission bandwidth and Frequency Planning due to UMTS utilising \emph{DS-CDMA} (direct sequence code division multiple access) and \emph{WCDMA} (Wide Band Code Division Multiple Access). The higher data transmission speed (2 Mb/s) can be attributed to UMTS using the DS-CDMA scheme. The scheme also allows more users to be served than previous generation of networks. A direct consequence of WCDMA which sends information over a wide-band of 5 MHz is that no frequency planning problem comparable to that of GSM has to be solved\cite{tabuglobalplanning3g,Eisenblatter}.

In this section we gave a brief overview of the history of the GSM Network specification. In the next section we will present an explanation of the topology of GSM network as well as look at the 
underlying problems present in a GSM networks.
\subsection{Topology of a GSM Network}
GSM networks consists of a variety of different subsystems to realise the goal of establishing a radio communication link between two parties. The hierarchy of systems and their respective connections to
each other is illustrated in figure 1. We will now briefly explain each subsystem.
\subsubsection{Mobile Station (MS)}
A Mobile station (MS) as it is defined in the GSM spec refers to any mobile device that is capable of of making and receiving calls on a GSM network.  The MS is the main gateway 
for a user to gain access to the GSM network. The MS has two features which play an important role throughout the GSM Network, namely:
\begin{description}
\item[Subscriber Identification Module (SIM)] --- Usually inserted into a mobile devices. The SIM contains the \emph{International Mobile Subscriber Identity} (IMSI) and is used throughout the network 
for Authentication as well as being a key part in providing encrypted transmissions.
\item[International Mobile Equipment Identity(IMEI)] --- Used to identify mobile station equipment. Primarily used in the denial of service to equipment that has been blacklisted\footnote{Equipment can be blacklisted for a variety of reasons e.g. theft} and tries to gain access to the network.
\end{description}
The MS has the capability to change the transmission power is uses from its base value to a maximum value of 20 mW. The change in transmission power is automatically set to the lowest level by the Base Transceiver Station to ensure reliable communication after evaluating the signal strength as measured by the MS \cite{GSMSysEngin}. % The power adjustment also minimizes cochannel interference, which we can talk about later when we introduce interference

\subsubsection{Base Station Subsystem (BSS)}

According to the GSM Phase 2+ specification this system is viewed by the \emph{Mobile Switching Centre} (MSC) through an Abis radio interface as the system responsible for communicating with Mobile Stations in a particular location area. The BSS usually consists of one \emph{Base Station Controller} (BSC) with one or more \emph{Base Transceiver Stations} (BTS) which it controls. The communication link between the MSC and BSC is the called the A-interface and the communication link between the BSC and BTS is called the Abis interface. The definition of these communication interfaces is beyond the scope of this disseration, the interested reader is directed to the book \emph{GSM System Engineering} by Asha Mehrotra. A BTS has similar equipment to that of a Mobile Station. Both have transceivers, antennas and the necessary functions to perform radio communication. 

In a GSM network the Service Area (SA) is subdivided into Location Areas(LA's) which are then futher divided into smaller radio zones called cells \cite{GSMSecurInTeleNetwork}. A cell is served by only one BTS and is usually regarded to be in the center of a cell. Even though cells are modelled as being hexagons (See figure 3) the actual coverage area of a cell has no predefined regular shape. Futhermore a cell is divided into 1 to 3 service sectors and each sector is allocated an antenna/transceiver \cite{GSMSysEngin}. Depending on how many sectors are at a cell, the operating angle of the antennae needs to be adjusted accordingly to ensure 360 degree service. If there is only one sector an omni-directional antenna is used, otherwise the antennae operating angle are adjusted to $\frac{360\,^{\circ}}{n}$ where ${n}$ is the amount of antennae \cite{Eisenblatter}.

Each sector operates one or more elementary transceivers called TRX’s. The amount of TRX’s per sector is determined by the expected peak traffic demand that the cell must be able to handle. Each TRX can handle 7 to 8 communication links or calls in parallel except the first TRX, which handles fewer calls than normal due to it being responsible for transmitting cell organisation and protocol information \cite{Eisenblatter}. TRX’s are able to handle 7-8 calls in parallel due to the use of \emph{Frequency Division Multiplexing} (FDM) and \emph{Time Division Multiplexing} (TDM) schemes. TRX’s are assigned channels which enable them to provide conversion between digital traffic data on the network side and the radio communication between Mobile Stations and 
the GSM network. \cite{ACOvsEA,FAPOrientationModel}

\subsubsection{Mobile Switching Centre (MSC)}

The MSC is at the heart of cellular switching system and forms part of the \emph{Network Switching Subsystem} (NSS). The MSC is responsible for the setting up, routing and supervision of calls between GSM subscribers. The MSC has interfaces on the one side to communicate with GSM subscribers (through the BSS) and on the other it has interfaces to communicate with external networks. The MSC interfaces with external networks to utilise their superior capability in data transmission as well as for the signalling and communication with other GSM entities \cite{GSM92}. 

The most basic functions that an MSC is responsibile for in a network are the following \cite{wirelesstelcoMullet}:
\begin{itemize}
\item Voice call initialization,routing,control and supervision between subscribers.
\item Handover process between two cells.
\item Location updating.
\item MS authentication.
\item SMS delivery.
\item Charging and Accounting of services used by subscriber.
\item Notification of other network elements.
\item Administration input or ouput processing functions.
\end{itemize}

To achieve most of these functions the MSC has an integrated \emph{Visitor Location Register (VLR)} database that stores call setup information for any MS that is currently registered for service with the MSC \cite{GSM92,wirelesstelcoMullet}. 

The VLR retrieves this information from the \emph{Home Location Register (HLR)} which contains all the registered GSM subcriber information for the network. This information enables the MSC to quickly retrieve the nesaccery information to setup a call between two entites \cite{GSMSysEngin,GSMSecurInTeleNetwork}.

A requirement for being able to communicate with other network elements such as \emph{Public Switching Telephone Networks} (PSTN) is the ability to multiplex and demultiplex signals to and from such network elements. This operation is a necacity, since the incoming or outgoing connection bitrate from the source entity might either be to low or to high for the receiving entity.

A typical scenario where this operation proves vital is when a mobile subcriber makes a call to a subcriber on a PSTN. The connection bit rate needs to be changed at the MSC from a wireless connection bitrate to a bitrate suitable for transmission over a PSTN.

\subsubsection{Network Databases}
The HLR, AUC (Authentication Center) and EIR (Equipment Identity Register) are the 3 'back-end' databases which stores and provides information for the rest of the GSM Network. We will now briefly discuss what the purpose of each database is and its core functions.

\paragraph{Home Location Register (HLR)}
The HLR is a database that permenantly stores information pertaining to a given set of subscribers. The HLR needs to store a wide range of subcriber parameters because it is the reference source for anything GSM subscriber related in the network. 

Subscriber parameters that are stored in the database include: Billing information, routing information, identification numbers, authentication parameters,subscribed services. The following information is also stored but the information is of a temporary nature and can change at anytime: Current VLR and MSC the subscriber is registered with; Wheter the subscriber is roaming \cite{GSMSysEngin}.

\paragraph{Authentication Center (AUC)}
The Authentication center is the entity in the GSM network that performs security functions and thus stores information that enables it to provide secure over the air communication. The information that is stored contains authentication information as well as keys that are used in ciphering of information.

During an authentication procedure no ciphering key is ever transmitted over the air, instead a challenge is issued to the mobile who needs to be authenticated. This callenge requires the mobile station to provide the correct \emph{Signed Response}(SRES) with regard to the random number generated by the AUC. The random number and ciphering keys that are used change with each call that is made, thus an attacker would gain nothing by intercepting a key, since it will change with the next call \cite{GSMSysEngin}.

Each mobile that is registered in the HLR database needs to be authenticated and each call that is instansiated needs to retrieve keys from the AUC to establish a secure communication link. The AUC is sometimes included with HLR to allow for fast communication between the two entities \cite{GSMSysEngin}.

\paragraph{Equipment Identity Register (EIR)}
The EIR is a database that stores the IMEI numbers of all registered mobile equipment that has accessed the network. Only information about the mobile equipment is stored, nothing about the subscriber or call is stored in the database.

Typically there is only one EIR database per network and interfaces to the various HLR database contained in the network. The IMEI's are grouped into 3 categories: \emph{White List}, \emph{Black List} and the \emph{Gray List}. The White list contains only the IMEI numbers of valid MS's; the Black List stores the IMEI numbers of equipment that has been reported stolen and the Gray List stores the IMEI numbers of equipment that has some fault (faulty software, wrong make of equipment).

\subsubsection{GSM Network Management entities}
In a GSM network most of the elements that form part of and make the network function are often distributed in a wide geographical area to provide the best network converage for the customer. 

For a network to function properly and efficiently network engineers need to be kept up to date on the state of the network and be alerted if \emph{any} problems occur. For this purpose there exists two systems in the GSM network architecture that allows for this functionality required by network engineers. 

The one system is called the Operations and Management center which is responsible for centralized regional and lcoal operational and maintenance activities. The other system called the Network Management System unlike the OMC provides global and centralized managemenent for operations and maintenance of the network suppored by the OMCs \cite{GSMSysEngin}.

We will not discuss the OMC and NMC in a bit more detail where we'll define the most critical functions they peform.

\paragraph{Operational and Managemenet Center}
The OMC is capable of communicating with GSM entities using two protocols namely SS7 and X.25. The SS7 protocol is usually used when the OMC is communicating within the GSM network over short and medium distances. The X.25 protocol is used for large external data transfers. All communication where the OMC is involved typically occurs over fixed line networks and/or leased lines. The OMC is usually used for day to day operation of a network \cite{GSMSysEngin}.

The OMC has support for alarm handling. An alarm in a GSM network goes of whenever a predefined expected condition does occurs. Engineers are able to define the severity of an alarm which defines who and what is futher alerted when and if the alarm is escalated to a higher level \cite{GSMSysEngin}.

To give one an idea of when and why and alarm goes of consider the following scenario: The MSC controls a set of BSS systems. Now for some reason a certain region experiences a power blackout. All the BSS affected by the blackout switch over to reserve power if availble. A first alarm is sounded to let the engineers / network know that the BSS is using reserve power. When the BSS reserve power is depleted the MSC sounds an alarm letting the network know that a specific BSS cannot be contacted.

Typically in the above scenario the severity of the first alarm will be of a medium priority. The second alarm is much more serious and its severity will be of a high priority.

The OMC is also capable of fault management in the GSM network. The OMC is able to activate,deactivate,remove and restore a service manually or automatically of network devices. Various tests can be run as well as diagnostic information can be retrieved on the network devices to detect any current or future defects \cite{GSMSysEngin}.

\paragraph{Network Management Center}
The NMC is similar to the OMC but it is not restricted to only regional GSM entities as it is in charge of the all GSM entities in the network. The NMC provides traffic management for the global network and also monitors high priority alarms such as overloaded or failed network nodes. It is usually used in long term planning of a network, but it has the capability to perform certain OMC functions when an OMC is not staffed. 

\subsection{GSM Network problems}
\section{The Frequency Assignment Problem}
The Frequency Assignment Problem (FAP) is a generalisation of the graph-colouring problem and is subsequently an NP-hard problem. This is because one has a finite amount of frequencies which needs to be 
assigned to antennae/transceivers (TRX's)  where the amount of transceivers to be assigned frequencies greatly out weigh the amount of available frequencies. Thus it is inevitable that a network will 
have interference and we can only minimise the amount of interference that might occur on the network - an optimisation problem. 

A contributing factor to the difficulty of the FAP is due to the scarcity of usable frequencies in the radio spectrum, which forces network operators to reuse their allocated/licensed frequencies in their respective networks. The scarcity of the usable frequencies in the spectrum can be attributed to the overuse of certain bands as well as large scale reuse of frequencies in networks. This has put strain on the spectrum and has complicated the management of networks significantly because interference is more likely to occur.

\subsection{Interference}
Interference occurs when frequencies assigned to connections differ by a small margin. The amount of inference on a connection defines the quality of service. One can naturally make the deduction that 
the more frequencies differ used on connections in a area, the better quality of service one will experience in that area. Cellular networks use the amount of interference on their networks as 
qualitative measure for their \emph{Quality of Service} (QoS). A network with high interference would experience a lot of dropped connections, which occurs when the interference is too high to sustain a connection for communication.

Primarily Interference occurs if the Electromagnetic constraints are violated, which are defined as:
\begin{description}
\item[Co-Cell] ---
\item[Adj Channel] ---
\item[Co-Site] ---
\end{description}
A fourth constraint, namely the Handover constraint is also applicable in Cellular networks which we will discuss in section 4.

\subsection{Frequency Assignment Types}
Within the FAP domain there are  a variety of sub problems which originated over the decades of which wireless communication has survived through. We will discuss the most popular problems found in the literature over the last few years in Section 2. The FAP can be classified into two categories:
\begin{enumerate}[\bf{(}a\bf{)}]
\item \emph{Fixed Frequency/Channel Assignment} (FFA/FCA) is the process of permanently assigning frequencies to cells (cellular towers). The frequencies assigned are fixed and cannot be changed on the fly while 
the network is active , since the frequencies assigned to the cell form part of a delicate frequency plan designed to keep interference to minimum.
\item \emph{Dynamic Frequency/Channel Assignment} (DFA/DCA) is the process of allocating channels to cells as they require it to meet the current traffic demand imposed on them by clients. 
\end{enumerate}
Each cell can be assigned multiple frequencies based on the amount of transmitters or TRX's it has. The amount of TRX's in a cell depends on the expected amount of traffic the particular cell must 
handle.

Most of the research in the FAP has concentrated on the FFA. The reason for this is because FFA is a static technique, which allows it to come up with a better solution since it has more time for 
calculation. FFA is also easier to implement in practise and allows the network operators to cater for the worst case scenario - heavy traffic load on the network. The DFA is at the moment a very hard 
problem because the network frequency plan is constantly changing, which means as the traffic on the network increases the longer the DFA focused algorithm will take to allocate a frequency. This 
increase in processing time is because the algorithm has to take into account more constraints with a lower available frequency pool. DFA must do this process within seconds since a cell needs to serve clients. Most researchers have concentrated on solving the FFA using heuristic approaches like neural networks, local search techniques and more recently meta heuristic approaches which include genetic algorithms, simulated annealing , ant colony optimisation and particle swarm optimisation.

In this section we have given a description of the Frequency Assignment Problem and introduced some concepts which we will use throughout the dissertation. In the next section we will present the 
Mathematics that govern the Frequency Assignment Problem.
\subsection{Mathematical Formulation}
\subsection{Types of Frequency Assignment Problems}
\subsubsection{Minimum Order Frequency Assignment Problem}
\subsubsection{Minimum Span (MS) Frequency Assignment Problem}
\subsubsection{Fixed Spectrum (FS) Frequency Assignment Problem}
\subsection{Frequency Assignment Models}
\subsubsection{Binary Constraints}
\subsubsection{Cost Function Minimisation}
\section{Summary}
