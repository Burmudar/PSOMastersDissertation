\chapter{Research Methodology}
\section{Introduction}
\section{Methodology}
This dissertation presents an algorithm that operates on a cellular phone network to optimise the allocation of frequencies used for communication in the network.
This section presents an introduction to the dissertation. The main themes of the dissertation are outlined namely, cellular phone networks and optimisation algorithms. The next section presents a research plan as well as chapter breakdown which is followed by this dissertation.
This study presents an algorithm that is based on the \gls{PSO} algorithm which is applied to the \gls{FS-FAP}. The purpose of creating such an algorithm is to determine the applicability of modern swarm-based algorithms with regard to finding better solutions to real-world problems that exist in domains such as cellular communication.

The research approach followed in this dissertation is first and foremost to understand the cellular network domain. More importantly, the focus is on how and what frequencies are used to facilitate communication as well as what affects the quality of the communicational link between two components in a cellular network 

The cellular domain is the problem domain, but to develop a solution in the problem domain a study needs to be conducted in the artificial intelligence domain as well. In order for a new optimisation algorithm to be developed, other algorithms that have achieved success in the respective optimisation problem domains in which they have been applied need to be investigated.

Therefore in an attempt to develop and apply a modern viable optimisation-based algorithm, a series of research questions has been identified which need to be answered. The research questions are as follows:
\section{Hypothesis}
\section{Research Questions}
\begin{itemize}
\item \textbf{Question 1} --- What is cellular technology and what components are involved when devices communicate in the network ?
\item \textbf{Question 2} --- What factors influence the quality of communication and what is interference ?
\item \textbf{Question 3} --- What exactly is the frequency problem and how does it affect modern mobile communication?
\item \textbf{Question 4} --- What are optimisation algorithms and what characteristics make them unique?
\item \textbf{Question 6} --- With regard to particle swarms, how can a particle best be represented as a frequency plan ?
\item \textbf{Question 7} --- With regard to particle swarms, how can one frequency plan be moved towards another frequency plan ?
\item \textbf{Question 8} --- With regard to particle swarms, how can particles be prevented from using forbidden frequencies when they move towards a particular plan ?
\end{itemize}

Throughout the course of this dissertation the aim is to answer each of these identified research questions. The next section lists a chapter breakdown of this dissertation.

\section{Measurements}
\section{Summary}

