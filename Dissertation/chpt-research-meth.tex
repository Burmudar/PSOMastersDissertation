\chapter{Research Methodology}
\section{Introduction}
This dissertation presents an algorithm that operates on a cellular phone network to optimise the allocation of frequencies used for communication in the network. In this chapter an outline will be given on the process that is followed by this dissertation for the research presented.

The chapter is structured as follows. The next section discusses the specific research methodology that is followed. The hypothesis which this research aims to investigate is then presented. After the hypothesis, a series of research questions are presented. The research presented in this dissertation aims to answer these questions.
Following the research questions a section on how the presented research will be measured and against what it will be measured. Finally the chapter will conclude with a summary.
\section{Methodology}
The research presented in the dissertation follows a combined approach between applied research and empirical research. 

Applied research is a good fit since it is concerned with an immediate problem that is faced by an organisation / business / industry or society. As stated in the introduction with the problem statement, the particular problem this research is about is the \gls{FAP} which is a problem experienced by wireless communication businesses. A more exhaustive description of the \gls{FAP} will be presented in chapter~\ref{chpt:fap}.

Part of the applied research is to better understand the domain of the problem as well as what exactly \gls{FAP} problem is. This research is presented in chapter~\ref{chpt:celltech} and chapter~\ref{chpt:fap}.

When the problem is understood, a study needs to be presented on other algorithms being applied to similar problems. This needs to be done in order to get a better understanding of the type of techniques that is required to make algorithms successfully operate on these type of problems.

The empirical research part to the research presented in this dissertation comes into affect with the main purpose of this dissertation which is to observe and experiment the effectiveness of applying the \gls{PSO} to the \gls{FAP}. The effectiveness of the algorithm presented in this research will be measured using the benchmark \gls{COST}. The benchmark closely resembles real world scenarios, more will be discussed about this benchmark in chapter~\ref{chpt:fap}.

The results of the algorithm operating on the \gls{COST} benchmark will be presented in chapter~\ref{chpt:results}. 
\section{Hypothesis}
The hypothesis for this research is to discern two key factors. The first factor, is it possible the apply the \gls{PSO} algorithm on the \gls{FAP}. It is no coincidence that the first factor is also the problem statement for this dissertation.

The second factor is, if the first factor is successful, to gauge the quality of the solutions that the algorithm produces.

The hypothesis presented are the main themes for the research. In the next section a series of research questions is presented.
\section{Research Questions}
\begin{itemize}
\item \textbf{Question 1} --- What is cellular technology and what components are involved when devices communicate in the network ?
\item \textbf{Question 2} --- What factors influence the quality of communication and what is interference ?
\item \textbf{Question 3} --- What exactly is the frequency problem and how does it affect modern mobile communication?
\item \textbf{Question 4} --- What are optimisation algorithms and what characteristics make them unique?
\item \textbf{Question 5} --- With regard to particle swarms, how can a particle best be represented as a frequency plan ?
\item \textbf{Question 6} --- With regard to particle swarms, how can one frequency plan be moved towards another frequency plan ?
\item \textbf{Question 7} --- With regard to particle swarms, how can particles be prevented from using forbidden frequencies when they move towards a particular plan ?
\end{itemize}

Throughout the course of this dissertation the aim is to answer each of these identified research questions. 

\section{Measurements}
As discussed previously, the algorithm will be compared against the best produced results for the \gls{COST} 259 benchmark suite. Specifically the siemens suite of benchmark instances. Chapter~\ref{chpt:fap} discusses in more detail why the siemens instances were chosen.

In addition to comparing the algorithm to the best \gls{COST} 259 results additional measurements is required. In particular, statistical detail on the frequency plan produced by the algorithm. The statistical detail helps in identifying where the frequency plan is generating too much interference and thus gives an outline where the algorithm needs to perform better optimisation.

The \gls{COST} 259 benchmarks also provide these interference statistics for all the results published, therefore it will also provide valuable insight where this dissertations algorithm needs to improve.

All these measurements are presented in chapter~\ref{chpt:results}
\section{Summary}
This chapter outlined the research methodology that is used to conduct research in this dissertation. The hypothesis was stated and a series of research questions which, when answered, will aid in proving the hypothesis.

Finally this chapter concluded with a discussion on the kind of measurements that will be required to assess the effectiveness of the algorithm presented in this dissertation. The next chapter provides a discussion on the cellular network domain.

