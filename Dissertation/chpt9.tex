\chapter{Conclusion}
\label{chpt:conclusion}
\section{Introduction}
The aim of this dissertation was to develop a modern artificial intelligence optimisation algorithm, which would be applied to a problem in the cellular technology domain. This aim was achieved: an algorithm was developed based on the standard \gls{PSO} algorithm which was then applied to operate on the FAP. 

In current research no similar PSO algorithm to date has been presented that has been applied to the FS-FAP using interference as a performance measurement. In chapter~\ref{chpt:psoapplicationFAP} the innovation and new techniques that made it possible for an algorithm based on the PSO to operate in the FS-FAP domain were discussed.

\section{Research Questions}
At the start of the dissertation a series of research questions were outlined. These questions were identified and served as a roadmap to understand the problem domain as well as to gather enough background information on optimisation algorithmic techniques to allow innovation.

Seven questions were identified in total. Each original question will now be listed together with a short discussion on how the question was answered and to which chapter it is related to.
\begin{itemize}
\item \textbf{Question 1} --- \emph{What is cellular technology and what components are involved when devices communicate in the network ?} This question was answered in chapter 2, which provided the history of cellular technology and also highlighted improvements that have been made since the inception of cellular networks.
\item \textbf{Question 2} --- \emph{What factors influence the quality of communication and what is interference ?} This question was answered in chapter 2 where the architecture of a modern-day cellular network was outlined and discussed in depth. Each entity used by the network to facilitate communication was identified and discussed in detail.
\item \textbf{Question 3} --- \emph{What exactly is the frequency problem and how does it affect modern wireless communication? } This question was answered in chapter 3, which contains a description of the general problem of the frequency assignment problem as well as exactly what happens in a cellular network for this problem to occur.
\item \textbf{Question 4} --- \emph{What are the most popular optimisation algorithms and what characteristics make them unique?} This question was answered in two chapters namely chapters 4 and 5. In these chapters the most popular and modern optimisation algorithms were evaluated.
\item \textbf{Question 5} --- \emph{With regard to particle swarms, how can a particle best be represented as a frequency plan ?} Chapter 6 described in detail how a frequency plan is represented as a particle in the algorithm.
\item \textbf{Question 6} --- \emph{With regard to particle swarms, how can one frequency plan be moved towards another frequency plan ?} Chapter 6 also describes the two methods that were developed to facilitate moving one frequency plan to another.
\item \textbf{Question 7} --- \emph{With regard to particle swarms, how can particles be prevented from using forbidden frequencies when they move towards a particular plan ?} Finally, chapter 6 also provided detail on exactly how when moving particles the algorithm avoids assigning invalid or forbidden frequencies.
\end{itemize}

\section{Proving the hypothesis}
In chapter 2, the hypothesis which this research aims to prove was stated. The hypothesis consisted of two key factors which needed to be proven. 

The first factor, was to discern whether it is possible to apply the \gls{PSO} to the \gls{FAP}. Based on the results presented in chapter~\ref{chpt:results} it has been proven that it is indeed possible to apply the PSO to the FAP.

The second factor was to gauge the quality of the solutions the a PSO operating on the FAP produces. As discussed in chapter~\ref{chpt:results} the FAP PSO algorithm does not produce high quality results when compared to the best results obtained on the COST 259 benchmark. This is understanable due to applying the PSO the specific FAP variant has never been attempted before. The base of a PSO algorithm operating the FAP has now been created. Therefore, further refinement on the algorithm is now possible and allows for experimentation with other methods. Possible methods are discussed in section~\ref{sec:futurework}.

The next section discusses the difficulties that were faced during the design and implementation of the FAP PSO algorithm. 
\section{Difficulties faced}
In the development of the PSO FAP algorithm a few challenges were faced. First and foremost there is no formal definition of the COST 259 fitness function. By just blindly summing all the interference values the calculated value will be wrong. One way to verify whether the interference calculation is corrected is to download one of the assignment plans that have been submitted. Each plan defines the total interference that will be generated by applying the plan. 

The key in calculating the right interference value with the COST 259 benchmark is to notice that a minimum tolerable interference value is defined in each and every problem. Therefore, if the generated interference is either less than or equal to this minimum value, the interference is tolerable and can be safely ignored.

Another difficulty that was faced was improving the efficiency of the algorithm. By using the parallel extensions framework in .Net 4 the algorithm uses all cores available, but at the expense of potential expensive context switching. Context switching especially becomes expensive when large populations are used as values need to be moved in and out of a cores cache as the threads move from core to core based on how they are scheduled.

In this section an overview of the difficulties that were encountered during the development of the FAP were discussed. In the following section an overview of the potential future work based on this study is presented.
\section{Future Work}
Most of the techniques developed for the FAP PSO, aim to stay true to the standard PSO algorithm. The next step is to hybridise the FAP PSO algorithm. A good candidate for hybridisation would be the GA as the algorithm naturally ``purifies'' results in finding the right genes that make up a good possible solution.

In the PSO a good entry point for the GA to be incorporated would be twofold. The first point would be to take a certain global best selected with the standard procedure and then mate it with successive global bests, with the offspring global best being a best for a future iteration. With this method the PSO is able to use the history of the algorithm from the start.

The second method takes a swarm of a certain iteration and then mates all the particles with each other for a certain number of iterations. This method can be seen as a means to clean the swarm of inefficient genes and could serve as an intensification phase for the algorithm.

Another point of interest is to disregard the PSO and rather try and produce an ABC algorithm on the FS-FAP. As mentioned, the ABC algorithm was not chosen for this research, since it is new and has not been applied to a wide variety of problems. 

By using the techniques developed in this research for instance the selection schemes, a viable ABC algorithm could be developed. The ABC algorithm is not that specific as to how new solutions are generated, as with the PSO which relies on vector mathematics, which allows an algorithm designer considerably more freedom.





