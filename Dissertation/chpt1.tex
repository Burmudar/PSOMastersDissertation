\chapter{Introduction}
\section{Introduction}
In the technology age life is almost unfathomable without the mobile phone. It is hard to believe how the business world manage to function in the pre-mobile phone era \footnote{Pre 1980's where the mobile phone didn't exist yet}. 

The invention of the mobile phone fullfilled the need of always being connected, always being within reach of the modern world. The need to be connected and within reach can actually be attributed to feel part of something and this something is deemed as being part of a network. Which isn't far off as to how mobile phones are able to provide the connectivity at almost any location within a country.

Behind the connectivity of mobile phones lies an intricate layer of communication technology which has evoled since the invention of the normal land line phone. This technology is reffered to at the top level as wireless communication, which includes inventions such as the radio. To enable communication on the level needed by mobile phones wirless communication had to evolve and go a step further than normal radio communcation. Hence the concept of Cellular mobile networks were developed.

Without cellular networks mobile phones have no means of providing communication. The purpose of the cellular network is the provide the nesacerry functions to connect and facilitate communication between two entities (mobile phones) over a wireless network. 

Cellular networks achieve this level of communication through expensive equipment and ``smart'' algorithms. By ``smart'' algorithms, what is actually meant, is algorithms that utilise aritificial intelligence concepts to perform a certain function in an attempt to either automate a task within the network or imporve a certain part of the network. Therefore, the correct term for these type of algorithms are actuall Artificial Intelligence (AI) Algorithms.

AI algorithms have a wide spread of functions that they can perform from making informed intelligent decisions to optimising the operation of certain processes. In particular computers are indeed very apt in optimising certain procedures. This is because a computer is able to test and evaluate a huge amount of different alterations and combinations of a certain procedure in a short amount of time, therefore allowing it to find the best combination out of those tested.

In this dissertation an algorithm will be presented which will concentrate on Cellular phone networks. The particular algorithm presented falls into the latter part of the AI algorithms discussed, namely an Aritificial Intelligence optimisation algorithm. Thus the algorithm presented in this dissertation will operate on a cellular phone network to optimise a certain part of the network.

In this section a introduction was presented to the dissertation. The main themes of the dissertation were outlined namely, cellular phone networks and optimisation algorithms. In the next section research plan and chapter breakdown will be presented which will be followed by this dissertation.
\section {Chapter breakdown}
\subsection{Part I - Background}
The first part of the dissertation is concerned with providing and discussing information in order to understand first of all the domain and problem the algorithm presented in this dissertation will address and finally to understand the intricisate details of how the presented algorithm operates.

Below is a outline of the chapters and the content they will discuss in the first part of this dissertation.
\subsubsection{Chapter 1}
This chapter is provides an introduction to the presented dissertation as well as providing a broad overview of the topics the dissertation will discuss. The chapter also provides a general outline of the chapters that will be presented in the dissertation.
\subsubsection{Chapter 2}
This chapter is concerned with providing information on how a mordern cellular network functions. Within this chapter a brief history is presented on how cellular network technology was developed. The chapter also provides a overview on the architecture of a cellular network discussing each part of the network intended purpose and function to facilitate wireless communication.
\subsubsection{Chapter 3}
This chapter presents the problem that the dissertation will address namely the Frequency Assignement Problem. The chapter provides a discussion on why the problem exists, what causes the problem to exist in the first place and what it means for a problem to be NP-Complete. Furthermore the chapter also discuss the various variants of the problem and how they differ depending on the wireless domain that is being considered. Finally the chapter also provides a formal definition of the problem which is later utilised by the algorithm developed in this dissertation.
\subsubsection{Chapter 4}
This chapter marks the beginning of a discussion on various optimisation algortihms in this dissertation. The algorithms presented in this chapter were chosen due to their wide spread usage as well as success on NP-Complete problems. The chapter discusses each algorithm in depth providing an outline of the core features that make the algorith unique as well as discussing each core feature in depth. For each algorithm the chapter also presents an analysis on related work of the particular algorithm being applied to the Frequency Assignment Problem. 
\subsubsection{Chapter 5}
This chapter is concerned with providing algorithms that are new in the research domain of optimisation algorithms. Where as the algorithms presented in chapter 4 are fairly old and have been applied to a wide variety of problems. Algorithms in this chapter are relatively new in the optimisation domain and have not been applied to nearly the same amount of problems as the algorithms in chapter 4. In this chapter swarm algorithms are presented and the algorithms have the particular characteristic that they are based generally on processes observed in nature. Each algorithm presented is discussed in depth with their core characteristics outlined and discussed. Furthermore for each algorithm a analysis is presented if the algorithm were to be applied to the Frequency Assignment Problem. Finally in this chapter it is also formally stated what algorithm this dissertation will apply to the Frequency Assignment Problem.
\subsection{Part II - Implementation}
\subsubsection{Chapter 6}
In this chapter various optimisation problems are presented. The problems presented were chosen due to their usage in the literature to test other optimisation algorithms. The problems are used to test two variants of the Particle Swarm Optimisation algorithm. The tests were done not only to determine the general viability of the algorithm but also to understand the algorithm beter. The chapter also presents the results obtained from applying the algorithms on the test problems together with a discussion on the performance obtained by the algorithms.
\subsubsection{Chapter 7}
This chapter is concerned with providing a discussion of the algorithm developed to be applied on the Frequency Assignment Problem. Within this chapter an outline is given as the process in developing a specialised particle swarm algorithm for the frequency algorithm. Each specialised technique developed is presented and discussed in depth as to why the technique is needed as well as why is it used by the algorithm.
\subsubsection{Chapter 8}
This chapter is concerned with providing the results of the algorithm developed and presented in chapter 7 after applying the algorithm to a specialised set of benchmark problems for Frequency Assignment Algorithms. The particular selected benchmark problems are discussed in chapter 2.
\subsubsection{Chapter 9}
This chapter is concerned with providing a conclusion to this dissertation. In this chapter it is determined whether the research goal was reached as well as whether there are any future work to improve the presented algorithm.
