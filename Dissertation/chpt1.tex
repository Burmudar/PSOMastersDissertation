\chapter{Introduction}
\section{Introduction}
In the technology age, life is almost unfathomable without the mobile phone. It is hard to believe how the business world managed to function in the pre-mobile phone era.

The invention of the mobile phone fulfilled the need to always be connected and be within reach of the modern world. This need can actually be attributed to feeling part of something and this something is deemed as being part of a network. This is similar to how mobile phones are able to provide connectivity at almost any location within a country.

Cellular networks provide mobile phones with the means to facilitate communication. Cellular networks achieve this level of communication with the use of expensive equipment and \gls{AI} algorithms.

\gls{AI} algorithms have a wide spread of functions that they can perform ranging from making informed intelligent decisions to optimising the operation of processes. In particular computers are indeed very apt at optimising procedures. This is because a computer is able to test and evaluate a huge number of different alterations and combinations of a procedure in a short amount of time, thereby allowing it to find the best combination out of those tested.

\section{Problem Statement}

\section {Chapter Breakdown}
\subsection{Part I - Background on the problem domain and influential algorithms}
The first part of this dissertation is concerned with the domain and problem the algorithm presented in this dissertation addresses, and finally to also understand the intricate details of how the algorithm operates.
Below is an outline of the chapters in the first part of this dissertation.
\subsubsection{Chapter 1}
This chapter provides an introduction to the dissertation as well as a broad overview of the topics that are discussed in this dissertation.
\subsubsection{Chapter 2}
This chapter defines the research methodology that is utilised through out the entire dissertation. Within the chapter a discussion is presented on the research approach followed, the hypothesis is defined and a plan is presented with which the hypothesis will be tested with.
\subsubsection{Chapter 3}
This chapter is concerned with providing information on how a modern cellular network functions. Within this chapter a brief history is presented on how cellular network technology was developed. The chapter also provides an overview of the architecture of a cellular network,and each part of the network's intended purpose and function to facilitate wireless communication is discussed.
\subsubsection{Chapter 4}
This chapter presents the problem that the dissertation addresses, namely the frequency assignment problem. The chapter provides a discussion on why the problem exists, the causes of the problem and what it means for a problem to be NP-Complete. Furthermore the variants of the problem and how they differ depending on the mobile telecommunications domain that is being considered are also discussed. Finally the chapter also provides a formal definition of the problem which is later utilised by the algorithm developed in this research.
\subsubsection{Chapter 5}
This chapter marks the beginning of a discussion on various optimisation algorithms in this dissertation. Each algorithm is discussed in depth providing an outline of the core features that make the algorithm unique as well as each core feature in detail. For each algorithm the chapter also presents an analysis on related work of the particular algorithm being applied to the frequency assignment problem. 
\subsubsection{Chapter 6}
This chapter is concerned with providing algorithms that are new in the research domain of swarm intelligence optimisation algorithms. In this chapter swarm algorithms are presented and the algorithms have the particular characteristic that they are based generally on processes observed in nature. Each algorithm is discussed in depth with its core characteristics outlined. Furthermore for each algorithm an analysis is given of the algorithm was to be applied to the frequency assignment problem.
\subsection{Part II - Discussion and results of implementing a PSO algorithm on the FAP}
\subsubsection{Chapter 7}
This chapter provides a discussion of the algorithm developed to be applied to the frequency assignment problem. Within this chapter an outline is given of the process in developing a specialised particle swarm algorithm for the frequency algorithm. Each specialised technique developed is discussed in depth, along with an explanation of why the technique is needed as well as why it is used by the algorithm.
\subsubsection{Chapter 8}
This chapter is concerned with providing the results after applying the algorithm to a specialised set of benchmark problems for frequency assignment algorithms. The particular selected benchmark problems were discussed in chapter 2.
\subsubsection{Chapter 9}
This chapter concludes this dissertation. In this chapter it is determined whether the research goal was reached as well as whether any future work can be done to improve the presented algorithm.

Note that the particular algorithms discussed in chapters four and five, were chosen as they influenced the development of the algorithm presented in this study. Chapter 6, where applicable, refers to how these algorithms have influenced the development of a technique used by the developed algorithm.
