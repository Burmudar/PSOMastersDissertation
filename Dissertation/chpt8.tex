\chapter{Results}
\label{chpt:results}
\section{Introduction}
The previous chapter presented a discussion on the FAP PSO algorithm . This algorithm was developed by modifying the standard PSO algorithm to operate on the FS-FAP. Thus far, no other PSO algorithms have been attempted on the FS-FAP. The only PSO algorithm that has been attempted in the FAP domain was discussed in chapter~\ref{chpt:swarm} and was not relevant to the study in this dissertation, as the PSO was applied to an entirely different FAP variant (MS-FAP). The PSO on the MS-FAP was not relevant because the performance measures and what the algorithms optimise differ.

With the FS-FAP, the main performance measurement is interference and the PSO aims to allocate frequencies in an optimal way to internally produce a frequency plan. On the other hand, the MS-FAP is concerned with the span of frequencies used and the performance measurement is based on the calls dropped. The main purpose of the PSO in the MS-FAP is to minimise the span of frequencies used and keep the number of dropped calls to a minimum.

The previous chapter discussed all the modifications that were made to the standard PSO. The modifications were made to enable PSO to operate on the FAP. Two velocity methods were developed and in addition to the standard global best selection scheme, two additional global best selection schemes were put forward. The algorithm presented in the previous chapter was benchmarked against the COST 259 benchmark Siemens instances.

This chapter presents the results of applying the FAP PSO algorithm with its different velocity methods as well as different global selection schemes. This is followed by how the different velocity methods affect the PSO performance as well as how the global selection schemes have an influence on the final results.

\section{PSO COST 259 Siemens Results}
The FAP PSO is applied to the benchmarks on a machine with the following specifications and frameworks:
\begin{itemize}
\item 4 GB RAM
\item Windows 7
\item Intel Quad Core CPU
\item C\# using .Net 4 Framework with Parallel Extensions
\end{itemize}
As discussed, the FAP PSO is applied to Siemens 1, Siemens 2, Siemens 3 and Siemens 4 of the COST 259 benchmark suite. For more information about the nature of the benchmarks, the reader is directed to section \ref{sec:COST259}.

For each benchmark, 12 results are presented. The following changes were made to the FAP PSO to obtain 12 different results, one for each variant of the algorithm.
\begin{itemize}
\item The two velocity methods that were developed for the PSO are tested.
\item Each velocity method is also paired with one of the three global selection schemes namely, standard (traditional PSO global best selection), building Gbest with Cells and building a Gbest with TRXs.
\end{itemize}
The following values are used for the FAP PSO algorithm regardless of the variant being benchmarked. These values were chosen after a series of trail runs showed that these values produce the best results.
\begin{itemize}
\item The swarm size was set to 20, 50 and 100.
\item Inertia was set to 0.5.
\item Cognitive coefficient was set to 0.4.
\item Social coefficient was set to 0.5.
\item The stopping criteria was set to max iterations 50.
\end{itemize}
The velocity methods and global selection schemes were discussed in the previous chapter. To obtain the best representation for the quality of results produced and performance of the different methods used by the algorithm, the benchmark is executed 20 times for every variant of the algorithm and different population size. 

The developed variants of the FAP PSO are compared against the algorithms that have obtained the best published results on the \gls{COST} 259 benchmarks as discussed in chapter \ref{chpt:heuristic}. The sections these algorithms were discussed in are as follows: \textbf{Dynamic Tabu} section~\ref{sec:tabuonfap}, \textbf{k-thin FAP} section~\ref{sec:saonfap} and \textbf{GA}~\ref{sec:gaonfap}. Note the name this dissertation will use to refer to these algorithms in this chapter, is in bold face font.

Each section in this chapter presents results on a \gls{COST} 259 benchmark instance and uses the following layout.
\begin{itemize}
        \item A table with overall results for velocity method 1 and 2. Each table presents the global best selection, population, total interference, average, standard deviation and variance. The average, standard deviation and variance values depicted in this table, were calculated based on the minimum interference achieved across the 20 different algorithm runs.
        \item Within each table the lowest interference obtained is indicated in \textbf{bold}.
        \item A table indicating interference statistics from the overall best frequency plan. The table has the following columns:
            \begin{itemize}
                \item{\textbf{co-channel max}} --- The max co-channel interference generated
                \item{\textbf{co-channel avg}} --- The average co-channel interference generated
                \item{\textbf{co-channel std}} --- The standard deviation of the generated co-channel interference
                \item{\textbf{adj-channel max}} --- The max adjacent-channel interference generated
                \item{\textbf{adj-channel avg}} --- The average adjacent-channel interference generated
                \item{\textbf{adj-channel std}} --- the standard deviation of the generated adjacent interference
                \item{\textbf{TRX max}} --- The max interference generated by a TRX
                \item{\textbf{TRX avg}} --- The avg interference generated by a TRX
                \item{\textbf{TRX std}} --- The standard deviation generated by a TRX
            \end{itemize}
        \item A table indicating the amount of TRX pairs exceeding a certain interference threshold for the overall best frequency plan.
        \item A line graph indicating the progress of the algorithm from iteration 1 to 50 as it obtained the best overall frequency plan for velocity method 1 and method 2.
\end{itemize}

After the results are presented a discussion will follow about the results and what they represent and reveal about the performance of the FAP PSO. The next section presents the results obtained for the benchmark instance Siemens 1.
\subsection{Siemens 1}
\begin{table}[H]
\centering
	\begin{tabular}{cccccc}
	\toprule
    GBest selection & Population & Interference & Average & Std. Deviation & Variance \\
    \midrule
    Standard & 100 & 100.23 & 103.62 &   9.44 &   4.05\\
    Standard & 50 & 100.90 & 105.69 &  10.44 &   4.19\\
    Standard & 20 & 101.15 & 106.70 &  12.23 &   5.54\\
    GBestFromCells & 20 &  36.41 &  39.60 &   9.09 &   3.18\\
    GBestFromCells & 50 &  \textbf{35.19} &  38.72 &   7.84 &   2.36\\
    GBestFromCells & 100 &  35.27 &  38.82 &   8.80 &   3.52\\
    GBestFromTrxs & 20 & 106.23 & 113.91 &  16.82 &  10.48\\
    GBestFromTrxs & 50 & 107.33 & 114.42 &  16.15 &  10.03\\
    GBestFromTrxs & 100 & 109.80 & 114.94 &  14.61 &   9.70\\
    \midrule
    Dynamic Tabu & --- & N/A & --- & --- & ---\\
    k-Thin SA & --- & 2.20 & --- & --- \\
    GA & --- & 2.96 & --- & --- \\
    \bottomrule
	\end{tabular}
\caption{Overall Siemens 1 results with velocity method 1}
\label{tab:siem1m1}
\end{table}
\begin{table}[H]
\centering
	\begin{tabular}{cccccc}
	\toprule
    GBest selection & Population & Interference & Average & Std. Deviation & Variance \\
    \midrule
    Standard & 20 & 378.53 & 784.02 & 794.81 & 23397.43\\
    Standard & 50 & 353.06 & 652.72 & 728.49 & 20411.46\\
    Standard & 100 & 317.16 & 511.48 & 613.45 & 17105.75\\
    GBestFromCells & 20 & 325.69 & 553.86 & 561.71 & 11685.80\\
    GBestFromCells & 50 & 161.25 & 322.53 & 373.15 & 5355.40\\
    GBestFromCells & 100 & 209.01 & 249.04 & 149.56 & 1016.78\\
    GBestFromTrxs & 20 & 193.83 & 577.81 & 1111.87 & 45787.46\\
    GBestFromTrxs & 50 & 149.19 & 347.40 & 885.59 & 30164.26\\
    GBestFromTrxs & 100 & \textbf{142.19} & 244.64 & 417.93 & 7939.47\\
    \midrule
    Dynamic Tabu & --- & \small{N/A} & --- & --- \\
    k-Thin SA & --- & 2.20 & --- & --- \\
    GA & --- & 2.96 & --- & --- \\
    \bottomrule
	\end{tabular}
\caption{Overall Siemens 1 results with velocity method 2}
\label{tab:siem1m2}
\end{table}
\begin{table}[H]
\centering
	\begin{tabular}{cccccccccc}
	\toprule
    Algorithm & \multicolumn{3}{c}{co-channel} & \multicolumn{3}{c}{adj-channel} & \multicolumn{3}{c}{TRX}\\
              & max & avg & std
              & max & avg & std
              & max & avg & std\\
    \midrule
    Best FAP PSO & 0.31 & 0.02 & 0.03 & 0.19 & 0.02 & 0.02 & 0.31 & 0.03 & 0.03\\ 
    Dynamic Tabu & \scriptsize{N/A} & \scriptsize{N/A} & \scriptsize{N/A} & \scriptsize{N/A} & \scriptsize{N/A} & \scriptsize{N/A} & \scriptsize{N/A} & \scriptsize{N/A} & \scriptsize{N/A}\\
    k-Thin SA & 0.03 & 0.00 & 0.00 & 0.03 & 0.00 & 0.00 & 0.05 & 0.01 & 0.01\\
    GA & \scriptsize{N/A} & \scriptsize{N/A} &  \scriptsize{N/A} &  \scriptsize{N/A} &  \scriptsize{N/A} &  \scriptsize{N/A} &  \scriptsize{N/A} &  \scriptsize{N/A} & \scriptsize{N/A}\\
    \bottomrule
	\end{tabular}
\caption{co-,adj-channel and TRX interference statistics for best Siemens 1 frequency plan}
\label{tab:stats-siem1m1}
\end{table}
\begin{table}[H]
\centering
	\begin{tabular}{cccccccccc}
	\toprule
    Algorithm & \multicolumn{9}{c}{TRX pairs exceeding interference}\\
    & 0.01 & 0.02 & 0.03 & 0.04 & 0.05 & 0.10 & 0.15 & 0.20 & 0.50 \\
    \midrule
    Best FAP PSO & 286 & 131 & 98 & 64 & 147 & 31 & 13 & 8 & 0\\
    Dynamic Tabu & \scriptsize{N/A} & \scriptsize{N/A} & \scriptsize{N/A} & \scriptsize{N/A} & \scriptsize{N/A} & \scriptsize{N/A} & \scriptsize{N/A} & \scriptsize{N/A} & \scriptsize{N/A}\\
    k-Thin SA & 33 & 4 & 1 & 0 & 0 & 0 & 0 & 0 & 0 \\
    GA & \scriptsize{N/A} & \scriptsize{N/A} & \scriptsize{N/A} & \scriptsize{N/A} & \scriptsize{N/A} & \scriptsize{N/A} & \scriptsize{N/A} & \scriptsize{N/A} & \scriptsize{N/A}\\
    \bottomrule
	\end{tabular}
\caption{TRX pair interference breakdown for best Siemens 1 frequency plan}
\label{tab:breakdown-siem1m1}
\end{table}

\subsection{Algorithm run graph for Siemens 1}
\begin{figure}[H]
	\begin{centering}
    \includegraphics[scale=0.50]{../Implementation/data-cruncher/graph/Siemens1.pdf}
	\caption{Algorithm run velocity method 1 versus method 2}
	\label{fig:siem1graph}
	\end{centering}
\end{figure}

\subsection{Siemens 2}
\begin{table}[H]
\centering
	\begin{tabular}{cccccc}
	\toprule
    GBest selection & Population & Interference & Average & Std. Deviation & Variance \\
    \midrule
    Standard & 20 &  73.32 &  77.65 &   6.20 &   1.43\\
    Standard & 50 &  74.23 &  76.85 &   4.38 &   0.74\\
    Standard & 100 &  74.19 &  76.65 &   3.54 &   0.62\\
    GBestFromCells & 20 &  \textbf{52.63} &  55.55 &   7.93 &   2.33\\
    GBestFromCells & 50 &  53.44 &  55.50 &   5.79 &   1.29\\
    GBestFromCells & 100 &  52.78 &  54.38 &   4.10 &   0.84\\
    GBestFromTrxs & 20 &  77.45 &  81.46 &  12.29 &   5.59\\
    GBestFromTrxs & 50 &  78.00 &  81.04 &   8.08 &   2.51\\
    GBestFromTrxs & 100 &  77.10 &  81.24 &   7.67 &   2.94\\
    \midrule
    Dynamic Tabu & --- & 14.28 & --- & --- & --- \\
    k-Thin SA & --- & 14.27 & --- & ---  & ---\\
    GA & --- & 17.83 & --- & ---  & ---\\
    \bottomrule
	\end{tabular}
\caption{Overall Siemens 2 results with velocity method 1}
\label{tab:siem2m1}
\end{table}
\begin{table}[H]
\centering
	\begin{tabular}{cccccc}
	\toprule
    GBest selection & Population & Interference & Average & Std. Deviation & Variance \\
    \midrule
    Standard & 20 & 174.21 & 206.56 &  82.50 & 252.09\\
    Standard & 50 & 161.40 & 207.63 &  90.62 & 315.83\\
    Standard & 100 & 158.31 & 199.62 &  77.74 & 302.17\\
    GBestFromTrxs & 20 & 106.79 & 121.19 &  43.35 &  69.61\\
    GBestFromTrxs & 50 & \textbf{101.03} & 112.28 &  36.08 &  50.07\\
    GBestFromTrxs & 100 & 102.06 & 108.32 &  16.95 &  14.36\\
    GBestFromCells & 20 & 250.15 & 461.26 & 524.44 & 10186.49\\
    GBestFromCells & 50 & 245.19 & 344.08 & 301.00 & 3484.60\\
    GBestFromCells & 100 & 173.36 & 255.28 & 210.97 & 2225.35\\
    \midrule
    Dynamic Tabu & --- & 14.28 & --- & --- & --- \\
    k-Thin SA & --- & 14.27 & --- & --- & --- \\
    GA & --- & 17.83 & --- & --- & --- \\
    \bottomrule
	\end{tabular}
\caption{Overall Siemens 2 results with velocity method 2}
\label{tab:siem2m2}
\end{table}
\begin{table}[H]
\centering
	\begin{tabular}{cccccccccc}
	\toprule
    Algorithm & \multicolumn{3}{c}{co-channel} & \multicolumn{3}{c}{adj-channel} & \multicolumn{3}{c}{TRX}\\
              & max & avg & std
              & max & avg & std
              & max & avg & std\\
    \midrule
    Best FAP PSO & 0.23 & 0.01 & 0.02 & 0.02 & 0.00 & 0.00 & 0.23 & 0.03 & 0.04 \\
    Dynamic Tabu & 0.11 & 0.01 & 0.01 & 0.02 & 0.00 & 0.00 & 0.20 & 0.03 & 0.03\\
    k-Thin SA & 0.07 & 0.01 & 0.01 & 0.02 & 0.00 & 0.00 & 0.16 & 0.03 & 0.03\\
    GA & \scriptsize{N/A} & \scriptsize{N/A} & \scriptsize{N/A} & \scriptsize{N/A} & \scriptsize{N/A} & \scriptsize{N/A} & \scriptsize{N/A} & \scriptsize{N/A} & \scriptsize{N/A}\\
    \bottomrule
	\end{tabular}
\caption{co-,adj-channel and TRX interference statistics for best Siemens 2 frequency plan}
\label{tab:stats-siem2m1}
\end{table}
\begin{table}[H]
\centering
	\begin{tabular}{cccccccccc}
	\toprule
    Algorithm & \multicolumn{9}{c}{TRX pairs exceeding interference}\\
    & 0.01 & 0.02 & 0.03 & 0.04 & 0.05 & 0.10 & 0.15 & 0.20 & 0.50 \\
    \midrule
    Best FAP PSO & 589 & 208 & 1400 & 92 & 193 & 33 & 8 & 3 & 0\\
    Dynamic Tabu & 343 & 89 & 24 & 18 & 9 & 1 & 0 & 0 & 0\\
    k-Thin SA & 359 & 71 & 27 & 17 & 13 & 0 & 0 & 0 & 0\\
    GA & \scriptsize{N/A} & \scriptsize{N/A} & \scriptsize{N/A} & \scriptsize{N/A} & \scriptsize{N/A} & \scriptsize{N/A} & \scriptsize{N/A} & \scriptsize{N/A} & \scriptsize{N/A}\\
    \bottomrule
	\end{tabular}
\caption{TRX pair interference breakdown for best Siemens 2 frequency plan}
\label{tab:breakdown-siem2m1}
\end{table}

\subsection{Algorithm run graph for Siemens 2}
\begin{figure}[H]
	\begin{centering}
    \includegraphics[scale=0.5]{../Implementation/data-cruncher/graph/Siemens2.pdf}
	\caption{Algorithm run velocity method 1 versus method 2}
	\label{fig:siem2graph}
	\end{centering}
\end{figure}


\subsection{Siemens 3}
\begin{table}[H]
\centering
	\begin{tabular}{cccccc}
	\toprule
    GBest selection & Population & Interference & Average & Std. Deviation & Variance \\
    \midrule
    Standard & 20 & 133.47 & 139.18 &  11.77 &   5.54\\
    Standard & 50 & 132.81 & 136.65 &   9.67 &   4.07\\
    Standard & 100 & 131.79 & 135.43 &   8.06 &   3.61\\
    GBestFromCells & 20 &  \textbf{44.64} &  48.37 &   8.06 &   2.60\\
    GBestFromCells & 50 &  46.50 &  48.79 &   5.08 &   1.12\\
    GBestFromCells & 100 &  45.37 &  48.75 &   7.11 &   2.81\\
    GBestFromTrxs & 20 & 132.93 & 145.69 &  25.55 &  26.12\\
    GBestFromTrxs & 50 & 136.42 & 144.03 &  19.09 &  15.84\\
    GBestFromTrxs & 100 & 136.86 & 145.75 &  17.21 &  16.45\\
    \midrule
    Dynamic Tabu & --- & 5.19 & --- & --- & --- \\
    k-Thin SA & --- & 4.73 & --- & --- & --- \\
    GA & --- & 6.08 & --- & --- & --- \\
    \bottomrule
	\end{tabular}
\caption{Overall Siemens 3 results with velocity method 1}
\label{tab:siem3m1}
\end{table}
\begin{table}[H]
\centering
	\begin{tabular}{cccccc}
	\toprule
    GBest selection & Population & Interference & Average & Std. Deviation & Variance \\
    \midrule
    Standard & 20 & 834.88 & 1203.31 & 1037.77 & 43078.83\\
    Standard & 50 & 488.22 & 963.06 & 1439.96 & 90151.59\\
    Standard & 100 & 360.76 & 701.69 & 1064.67 & 62973.79\\
    GBestFromCells & 20 & 413.96 & 571.54 & 535.42 & 11467.16\\
    GBestFromCells & 50 & 269.23 & 392.32 & 319.11 & 4427.39\\
    GBestFromCells & 100 & 221.26 & 315.09 & 203.02 & 2289.75\\
    GBestFromTrxs & 20 & 229.05 & 619.35 & 1365.12 & 74541.80\\
    GBestFromTrxs & 50 & 222.19 & 370.61 & 646.37 & 18165.03\\
    GBestFromTrxs & 100 & \textbf{210.01} & 277.58 & 350.43 & 6822.20\\
    \midrule
    Dynamic Tabu & --- & 5.19 & --- & --- & --- \\
    k-Thin SA & --- & 4.73 & --- & --- & --- \\
    GA & --- & 6.08 & --- & --- & --- \\
    \bottomrule
	\end{tabular}
\caption{Overall Siemens 3 results with velocity method 2}
\label{tab:siem3m2}
\end{table}
\begin{table}[H]
\centering
	\begin{tabular}{cccccccccc}
	\toprule
    Algorithm & \multicolumn{3}{c}{co-channel} & \multicolumn{3}{c}{adj-channel} & \multicolumn{3}{c}{TRX}\\
              & max & avg & std
              & max & avg & std
              & max & avg & std\\
    \midrule
    Best FAP PSO & 0.22 & 0.01 & 0.02 & 0.14 & 0.00 & 0.01 & 0.22 & 0.03 & 0.02 \\
    Dynamic Tabu & 0.04 & 0.00 & 0.00 & 0.03 & 0.00 & 0.00 & 0.07 & 0.01 & 0.01\\
    k-Thin SA & 0.03 & 0.00 & 0.00 & 0.02 & 0.00 & 0.00 & 0.08 & 0.01 & 0.01\\
    GA & \scriptsize{N/A} & \scriptsize{N/A} & \scriptsize{N/A} & \scriptsize{N/A} & \scriptsize{N/A} & \scriptsize{N/A} & \scriptsize{N/A} & \scriptsize{N/A} & \scriptsize{N/A}\\
    \bottomrule
	\end{tabular}
\caption{co-,adj-channel and TRX interference statistics for best Siemens 3 frequency plan}
\label{tab:stats-siem3m1}
\end{table}
\begin{table}[H]
\centering
	\begin{tabular}{cccccccccc}
	\toprule
    Algorithm & \multicolumn{9}{c}{TRX pairs exceeding interference}\\
    & 0.01 & 0.02 & 0.03 & 0.04 & 0.05 & 0.10 & 0.15 & 0.20 & 0.50 \\
    \midrule
    Best FAP PSO & 494 & 228 & 150 & 91 & 146 & 27 & 6 & 4 & 0 \\
    Dynamic Tabu & 88 & 14 & 3 & 0 & 0 & 0 & 0 & 0 & 0\\
    k-Thin SA & 80 & 6 & 4 & 0 & 0 & 0 & 0 & 0 & 0\\
    GA & \scriptsize{N/A} & \scriptsize{N/A} & \scriptsize{N/A} & \scriptsize{N/A} & \scriptsize{N/A} & \scriptsize{N/A} & \scriptsize{N/A} & \scriptsize{N/A} & \scriptsize{N/A}\\
    \bottomrule
	\end{tabular}
\caption{TRX pair interference breakdown for best Siemens 3 frequency plan}
\label{tab:breakdown-siem3m1}
\end{table}

\subsection{Algorithm run graph for Siemens 3}
\begin{figure}[H]
	\begin{centering}
    \includegraphics[scale=0.5]{../Implementation/data-cruncher/graph/Siemens3.pdf}
	\caption{Algorithm run velocity method 1 versus method 2}
	\label{fig:siem3graph}
	\end{centering}
\end{figure}



\subsection{Siemens 4}
\begin{table}[H]
\centering
	\begin{tabular}{cccccc}
	\toprule
    GBest selection & Population & Interference & Average & Std. Deviation & Variance \\
    \midrule
    Standard & 20 & 513.86 & 527.40 &  28.30 &  30.81\\
    Standard & 50 & 507.81 & 521.39 &  25.28 &  29.05\\
    Standard & 100 & 512.97 & 517.03 &   9.02 &  10.17\\
    GBestFromCells & 20 & 284.60 & 291.73 &  26.44 &  26.89\\
    GBestFromCells & 50 & \textbf{277.36} & 288.84 &  22.52 &  23.06\\
    GBestFromCells & 100 & 282.36 & 288.14 &  15.41 &  33.94\\
    GBestFromTrxs & 20 & 526.47 & 535.88 &  31.79 &  38.87\\
    GBestFromTrxs & 50 & 518.61 & 533.59 &  35.25 &  56.49\\
    GBestFromTrxs & 100 & 523.24 & 533.67 &  13.59 &  23.07\\
    \midrule
    Dynamic Tabu & --- & 81.88 & --- & ---& --- \\
    k-Thin SA & --- & 77.25 & --- & ---& --- \\
    GA & --- & 96.84 & --- & ---& --- \\
    \bottomrule
	\end{tabular}
\caption{Overall Siemens 4 results with velocity method 1}
\label{tab:siem4m1}
\end{table}
\begin{table}[H]
\centering
	\begin{tabular}{cccccc}
	\toprule
    GBest selection & Population & Interference & Average & Std. Deviation & Variance \\
    \midrule
    Standard & 20 & 1482.01 & 2184.35 & 1232.51 & 58426.63\\
    Standard & 50 & 1646.26 & 2058.09 & 987.90 & 44361.28\\
    Standard & 100 & 1746.22 & 2036.60 & 366.09 & 16752.78\\
    GBestFromCells & 20 & 1176.55 & 1572.82 & 1208.47 & 56169.28\\
    GBestFromCells & 50 & 862.67 & 1078.10 & 702.90 & 22457.42\\
    GBestFromCells & 100 & 834.08 & 1007.29 & 286.25 & 10242.22\\
    GBestFromTrxs & 20 & \textbf{826.82} & 1761.64 & 2234.28 & 192000.61\\
    GBestFromTrxs & 50 & 942.28 & 1261.93 & 903.42 & 37098.69\\
    GBestFromTrxs & 100 & 844.45 & 1092.90 & 363.64 & 16529.03\\
    \midrule
    Dynamic Tabu & --- & 81.88 & --- & --- & --- \\
    k-Thin SA & --- & 77.25 & --- & ---  & ---\\
    GA & --- & 96.84 & --- & ---  & ---\\
    \bottomrule
	\end{tabular}
\caption{Overall Siemens 4 results with velocity method 2}
\label{tab:siem4m2}
\end{table}
\begin{table}[H]
\centering
	\begin{tabular}{cccccccccc}
	\toprule
    Algorithm & \multicolumn{3}{c}{co-channel} & \multicolumn{3}{c}{adj-channel} & \multicolumn{3}{c}{TRX}\\
              & max & avg & std
              & max & avg & std
              & max & avg & std\\
    \midrule
    Best FAP PSO & 0.57 & 0.01 & 0.03 & 0.08 & 0.00 & 0.00 & 0.57 & 0.08 & 0.07\\
    Dynamic Tabu & 0.20 & 0.01 & 0.01 & 0.05 & 0.00 & 0.00 & 0.43 & 0.06 & 0.05\\
    k-Thin SA & 0.19 & 0.01 & 0.01 & 0.05 & 0.00 & 0.00 & 0.36 & 0.06 & 0.05\\
    GA & \scriptsize{N/A} & \scriptsize{N/A} & \scriptsize{N/A} & \scriptsize{N/A} & \scriptsize{N/A} & \scriptsize{N/A} & \scriptsize{N/A} & \scriptsize{N/A} & \scriptsize{N/A}\\
    \bottomrule
	\end{tabular}
\caption{co-,adj-channel and TRX interference statistics for best Siemens 4 frequency plan}
\label{tab:stats-siem4m1}
\end{table}
\begin{table}[H]
\centering
	\begin{tabular}{cccccccccc}
	\toprule
    Algorithm & \multicolumn{9}{c}{TRX pairs exceeding interference}\\
    & 0.01 & 0.02 & 0.03 & 0.04 & 0.05 & 0.10 & 0.15 & 0.20 & 0.50 \\
    \midrule
    Best FAP PSO & 1970 & 889 & 452 & 286 & 666 & 253 & 127 & 247 & 5\\
    Tabu & 2161 & 971 & 547 & 344 & 209 & 12 & 2 & 0 & 0\\
    SA & 2053 & 871 & 445 & 282 & 163 & 11 & 2 & 0 & 0\\
    GA & \scriptsize{N/A} & \scriptsize{N/A} & \scriptsize{N/A} & \scriptsize{N/A} & \scriptsize{N/A} & \scriptsize{N/A} & \scriptsize{N/A} & \scriptsize{N/A}  \\
    \bottomrule
	\end{tabular}
\caption{TRX pair interference breakdown for best Siemens 4 frequency plan}
\label{tab:breakdown-siem4m1}
\end{table}
\subsection{Algorithm run graph for Siemens 4}
\begin{figure}[H]
	\begin{centering}
    \includegraphics[scale=0.5]{../Implementation/data-cruncher/graph/Siemens4.pdf}
	\caption{Algorithm run velocity method 1 versus method 2}
	\label{fig:siem4graph}
	\end{centering}
\end{figure}


\subsection{Comparison with best results in COST 259}
Below the best results achieved by the FAP PSO are compared to the best results obtained with each Siemens problem as published on the FAP website\cite{FAPWeb}.
\begin{table}[H]
\centering
\begin{tabular}{ccccc}
	\toprule
	Algorithm & Siemens 1 & Siemens 2 & Siemens 3 & Siemens 4 \\
    \midrule
	FAP PSO & 35.19 & 52.63 & 44.64 & 277.36 \\ 
	k-thin FAP & \textbf{2.20} & \textbf{14.27} & \textbf{4.73} & \textbf{77.25} \\ 
    Dynamic Tabu Search & \scriptsize{N/A} & 14.28 & 5.19 & 81.88 \\
	GA & 2.96 & 17.83 & 6.08 & 96.84 \\
    \bottomrule
	\end{tabular}
\caption{FAP PSO results compared to best obtained results}
\label{tab:allbest}
\end{table}

All the results and relevant statistics have now been presented. The section that follows, presents a discussion and analysis of the results.
\section{The Performance of the PSO}
In this section the effects of the changes made to the FAP PSO algorithm in terms of the results are discussed. First, the effect of the two developed velocity methods in terms of the results are described. 

In section~\ref{sec:diffglobalschemes} the three global schemes that are used by the algorithm are discussed. This section concludes with a discussion on the interference statistics and what it means for the peformance of the FAP PSO.
\subsection{Velocity Method 1 vs. Method 2}
The FAP PSO algorithm is able to utilise two different velocity methods to move the swarm around in the FAP problem space. The algorithms that implement these two methods were presented in section~\ref{sec:velocityFAP} and section~\ref{sec:velocityFAP2}.

By analysing the results, it becomes abundantly clear that velocity method 1 is by far the superior method for moving in the problem space. In each of the results, when comparing end fitness values produced by algorithm variants that use method 1 one can easily come to the conclusion that method 1 performs better than method 2.

As discussed in chapter~\ref{chpt:psoapplicationFAP}, method 1 uses a stage-based approach when applying the velocity function whereas method 2 applies the velocity function as is without it being broken up. Based on the results, using a stage-based approach to apply the velocity function is far better than applying the velocity equation directly to the transceiver in the plan.

Method 1 works by moving the whole swarm through to each stage before applying the next stage in the velocity equation. Thus after each stage, the whole swarm is at the same phase of the equation which keeps the swarm structured.

With method 2, the whole velocity equation is applied to whatever value is supposed to be operated on. Thus when method 2 moves a particle the frequency plan is moved piece by piece to a destination in the problem space. How method 2 accomplishes this movement was discussed in chapter~\ref{chpt:psoapplicationFAP} section~\ref{sec:velocityFAP2}.

By applying the velocity equation it is difficult to control the algorithm search process. With the FAP PSO control is necessary as there are various constraints that must not only be avoided but also adhered to for the generated plan to be usable. With method 2 adding domain knowledge is difficult, since after the velocity equation has been calculated the particle is very close to being moved to a new position. All that still needs to be done, before a particle is moved, is to apply inertia, which means there is a minor check that can be done to ensure that all the frequency values are within acceptable bounds.

By breaking the velocity equation up into smaller parts (stages) using method 1 the algorithm is able to direct and ensure that the swarm is moving in the general direction of valid frequency allocations.

Also the algorithm is able to embed domain knowledge earlier into the calculation of the velocity and is therefore able to intercept early on movements that will result in invalid frequency allocations at each stage of the velocity equation.

Taking into accounts how the two methods move particles around the solution space as well as the calculated standard deviation and variance a defining characteristic can be derived for method 1 and for method 2. Based on the standard deviation and variance, velocity method 1 is consistent and steadily moves to better positions. Velocity method 2 has high deviation and variance, which means that the algorithm explores the search space a lot more and is less directed towards a particular section of the solution space where good solutions might be. 

Velocity method 1 has the key characteristic of performing an intensification search which is similar to the intensification phase in \gls{TS} discussed in \ref{chpt:heuristic}. In contrast velocity method 2 has the key characteristic of diversifying and exploration, similar to the diversification phase of the \gls{TS} discussed in chapter ~\ref{chpt:heuristic}. These characteristics become even more evident when analysing each graph presented. In all the graphs, velocity method 1 starts of at a initial point and then steadily moves downward to better positions. Where as, velocity method 2 is very sporadic, moving up and down. 

Except in Siemens 2 where the algorithm was a lot less sporadic considering how sporadic the search is when compared to other instances of the method operating on benchmark instances. When only comparing the search process of velocity method 2 to 1 on the graph and taking into account the variance as well as the standard deviation, the characteristic of method 2 still holds. The search seems less sporadic, but based on the variance, which is a lot more than method ones variance, the algorithm with method 2 explores a lot more comparatively.

Based on the graphs, velocity method 2 can also perform the function of a disruptor and break the algorithm out of local minima.

\subsection{Different Global Schemes}
\label{sec:diffglobalschemes}
The previous chapter identified and discussed three global selection schemes. The first global selection scheme uses the standard PSO selection and the particle with the best fitness is the global best. This scheme is called ``Standard GBest''.

As discussed in section~\ref{sec:buildglobalbest}, using the standard gbest selection scheme is not preferable as it can lead to the swarm losing out on good frequency allocations due to overshadowing of frequencies\footnote{Overshadowing is discussed in chapter~\ref{chpt:psoapplicationFAP}.}.
Even with overshadowing the standard global scheme does not produce bad results, which seems to indicate that overshadowing of frequency allocations does not impact the frequency plans as significantly as thought initially.

In addition to the standard gbest selection scheme, two other selection schemes were tested. It is incorrect to call these schemes selection schemes of gbest, since they build global best rather than select them.

By far the worst performing scheme is where the global best is built from transceivers. In every benchmark performed where this scheme was paired with a velocity method and population size, the algorithm was simply not able to produce any good solutions. All possible solutions had high interference values, making them undesirable.

The bad performance of the build from transceivers scheme can be attributed to the granularity it uses to build a global best. As outlined in section~\ref{sec:buildglobalbest} the scheme only considers the interference generated by a single frequency allocated to a transceiver. This would have worked well if there were some sort of guarantee that a particular transceiver would only be interfered with by one other transceiver.

In reality and in the Siemens 4 benchmarks this is definitely not the case. More often than not, transceivers are interfered with by more than one other transceiver. Thus by only concentrating on a single case-by-case basis of frequencies allocated to transceivers, the scheme is discarding all other possible interferences. 

It might select a frequency at one point as the best, since in that scenario, the interference generated with the only other transceiver that is considered at that point is low. But this particular frequency is too close on the spectrum to another frequency allocated to some other transceiver that also interferes. Due to the algorithm only considering individual cases, this potential interference with the other transceiver will not be noticed by the algorithm and it will go ahead in selecting the frequency as the best for the transceiver.

By analysing the results produced by the various FAP PSO algorithms, it can be concluded that the best global best selection scheme is by far the one in which cells are used to build a global best. With the cell selection scheme, the algorithm does not suffer the pitfall that is the reason for the transceiver gbest building scheme's bad performance.

As discussed the build from cells scheme uses cells to build a gbest, and thus each cell stores the interference that the frequencies allocated to its transceivers generate by interfering with other cells. As a cell interferes with other cells, the interference generated is added to the cell causing the interference.

After the PSO has calculated the fitness of all positions, each cell will contain the interference it personally has caused throughout the network to other cells. A cell with low interference means the frequencies that have been allocated to this particular cell are the best combination that causes the least amount of interference. Therefore, with the build gbest from cells scheme, the algorithm is able to make informed choices when selecting a cell to be included in the global best. 
\subsection{The average, standard deviation and variance}
For each of the presented algorithm variants that are presented the average, standard deviation and variance was calculated. As mentioned in the introduction of this chapter, each algorithm variant was executed 20 times. The statistics presented are thus not based on a single run but on this set of 20 results for each variant. The minimum achieved is the lowest minimum among the 20 result set.

By analysing the statistics presented for each variant of the algorithm it can be concluded that by using velocity method 1, the swarm search process is more directed and focused. The deviation obtained while using velocity method 1 is low meaning that the gbest value of the swarm is near the average value of the gbest obtained across the swarm. Thus by using velocity method 1 the search process of the swarm is more focused and does not explore as much.

On the other hand, with the high deviation and high average and especially the high variance, it can be concluded that velocity method 2 pushes the swarm to explore the problem space a lot more. Even though the search space is explored more, with velocity method 2 the algorithm fails to intensify and converge on a good solution. This lack of convergence is evident if one analyses the algorithm progress graphs presented in fig. ~\ref{fig:siem1graph}, \ref{fig:siem2graph}, ~\ref{fig:siem3graph} and \ref{fig:siem4graph}. Note that these graphs are based on a single best produced result out of all the produced results by all the variants of the algorithm. The graphs establishes the notion clearly that velocity method 1 is more focused and directed and velocity method 2 explores, disrupts or diversifies.

Unfortunately the algorithms to which the FAP PSO is compared against namely, Dynamic Tabu, k-Thin SA and GA only published their respective minimums interference for their generated frequency plans along with the plans interference statistics. Therefore, in the next section, the interference statistics for best generated frequency plans by the FAP PSO for each benchmark is discussed.

\subsection{Interference statistics}
In the results sections two tables were presented in addition to the table depicting the overall algorithm performance. As discussed in the introduction, each algorithm variant was executed 20 times. The two additional tables, depicts the interference statistics of the single best result out of all the produced different algorithm results.

Based on the co-channel interference statistics for all of the benchmarks, the FAP PSO is prone to assign frequencies to TRXs which induce co-channel interference. This can be seen due to the average and standard deviation being higher than the adjacent-channel interference.

Potential co-channel interference is difficult to detect because in the algorithm, at the time of frequency assignment, each cell is not aware with which other cells it interferes. It is possible that a cell could interfere with all of the other cells in a network. If only a single frequency plan was being considered making the cell aware of the cells it interferes with would be optional but in the FAP PSO it isn't. Due to the FAP PSO using a population to explore a search space, making a cell aware would entail making it aware of cells in each and every single individual of the swarm. Even if one were to implement such awareness on a cell, it will complicate new frequency assignment and prolong processing by a huge amount since the whole populations cells need to be checked for co-channel violations.

It should also be noted, that the FAP PSO algorithm performs poorly in minimising the effect a co-channel assignment will have. This is evident since in all of the presented interference statistics, the max co-channel interference is also the highest TRX interference. Compared to the Dynamic Tabu and k-Thin the FAP PSO max co-channel violation is an order of magnitude more.

In contrast the adj-channel interference statistics compare favourably to the Dynamic Tabu and k-Thin averages and standard deviation. Again, as noted previously, the FAP PSO algorithm does not minimise the effect of a co- or adjacent channel interference. The max adjacent-channel interference is marginally higher than those reported by Dynamic Tabu and k-Thin.

When only considering the interference statistics of max, average and standard deviation, it is not clear as to why the FAP PSO algorithm generate frequency plans with such high interference when compared to the best frequency plans. Although when considering the interference statistics relating to the amount of TRXs exceeding a interference threshold, it becomes clear. The FAP PSO algorithm simply has far too many TRX pairs in the 0.10, 0.15, 0.20 and 0.50 thresholds. Comparatively the Dynamic Tabu and k-Thin typically have close to 10, but mostly 0 in these high interference thresholds. The FAP PSO on the other hand, has in just about every case above 100 TRX pairs that exceed the threshold.

The TRX interference statistics makes it abundantly clear where the FAP PSO needs to improve. For the algorithm to make any progress towards generated frequency plans with lower interference, the amount of TRXs that exceed the higher thresholds from 0.10 needs to be reduced drastically.

Further improvements that will require future work is discussed in the next chapter.

\section{Summary}
In this chapter the results produced by the algorithm discussed in chapter~\ref{chpt:psoapplicationFAP} were presented. The FAP PSO algorithm was applied to four COST 259 benchmarks namely Siemens 1, Siemens 2, Siemens 3 and Siemens 4. These four benchmarks were discussed in detail in chapter~\ref{chpt:fap}. For each of the benchmarks, 12 different variants of the FAP PSO algorithm were tested. Each variant used a different velocity function, global best selection scheme or population size. For each benchmark three tables were presented. The first table presented the results for the overall performance of the different algorithm variants on the benchmark. The second and third table presented the interference statistics for the best generated frequency plan by the FAP PSO. Finally each section concluded with a graph depicting the performance of the best FAP PSO algorithm run for velocity method 1 and velocity method 2.

The chapter concluded with critical analysis of each of the different algorithms developed in this study to enable the PSO to operate in the FAP space as well as a discussion on the interferences statistics and what it reveals about the FAP PSO performance.
