\chapter{The Frequency Assignment Problem}
\section{Introduction}
The Frequency Assignment Problem (FAP) is a generalisation of the graph-colouring problem and is subsequently an NP-hard problem. This is because one has a finite amount of frequencies which needs to be 
assigned to antennae/transceivers (TRX's)  where the amount of transceivers to be assigned frequencies greatly out weigh the amount of available frequencies. Thus it is inevitable that a network will 
have interference and we can only minimise the amount of interference that might occur on the network - an optimisation problem. 

A contributing factor to the difficulty of the FAP is due to the scarcity of usable frequencies in the radio spectrum, which forces network operators to reuse their allocated/licensed frequencies in their respective networks. The scarcity of the usable frequencies in the spectrum can be attributed to the overuse of certain bands as well as large scale reuse of frequencies in networks. This has put strain on the spectrum and has complicated the management of networks significantly because interference is more likely to occur.

\subsection{Interference}
Interference occurs when frequencies assigned to connections differ by a small margin. The amount of inference on a connection defines the quality of service. One can naturally make the deduction that 
the more frequencies differ used on connections in a area, the better quality of service one will experience in that area. Cellular networks use the amount of interference on their networks as 
qualitative measure for their \emph{Quality of Service} (QoS). A network with high interference would experience a lot of dropped connections, which occurs when the interference is too high to sustain a connection for communication.

Primarily Interference occurs if the Electromagnetic constraints are violated, which are defined as:
\begin{description}
\item[Co-Cell] ---
\item[Adj Channel] ---
\item[Co-Site] ---
\end{description}
A fourth constraint, namely the Handover constraint is also applicable in Cellular networks which we will discuss in section 4.

\subsection{Frequency Assignment Types}
Within the FAP domain there are  a variety of sub problems which originated over the decades of which wireless communication has survived through. We will discuss the most popular problems found in the literature over the last few years in Section 2. The FAP can be classified into two categories:
\begin{enumerate}[\bf{(}a\bf{)}]
\item \emph{Fixed Frequency/Channel Assignment} (FFA/FCA) is the process of permanently assigning frequencies to cells (cellular towers). The frequencies assigned are fixed and cannot be changed on the fly while 
the network is active , since the frequencies assigned to the cell form part of a delicate frequency plan designed to keep interference to minimum.
\item \emph{Dynamic Frequency/Channel Assignment} (DFA/DCA) is the process of allocating channels to cells as they require it to meet the current traffic demand imposed on them by clients. 
\end{enumerate}
Each cell can be assigned multiple frequencies based on the amount of transmitters or TRX's it has. The amount of TRX's in a cell depends on the expected amount of traffic the particular cell must 
handle.

Most of the research in the FAP has concentrated on the FFA. The reason for this is because FFA is a static technique, which allows it to come up with a better solution since it has more time for 
calculation. FFA is also easier to implement in practise and allows the network operators to cater for the worst case scenario - heavy traffic load on the network. The DFA is at the moment a very hard 
problem because the network frequency plan is constantly changing, which means as the traffic on the network increases the longer the DFA focused algorithm will take to allocate a frequency. This 
increase in processing time is because the algorithm has to take into account more constraints with a lower available frequency pool. DFA must do this process within seconds since a cell needs to serve clients. Most researchers have concentrated on solving the FFA using heuristic approaches like neural networks, local search techniques and more recently meta heuristic approaches which include genetic algorithms, simulated annealing , ant colony optimisation and particle swarm optimisation.

In this section we have given a description of the Frequency Assignment Problem and introduced some concepts which we will use throughout the dissertation. In the next section we will present the 
Mathematics that govern the Frequency Assignment Problem.
\subsection{Mathematical Formulation}
\subsection{Types of Frequency Assignment Problems}
\subsubsection{Minimum Order Frequency Assignment Problem}
\subsubsection{Minimum Span (MS) Frequency Assignment Problem}
\subsubsection{Fixed Spectrum (FS) Frequency Assignment Problem}
\subsection{Frequency Assignment Models}
\subsubsection{Binary Constraints}
\subsubsection{Cost Function Minimisation}
