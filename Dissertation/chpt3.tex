\chapter{The Frequency Assignment Problem}
\label{chpt:fap}
\section{Introduction}
The Frequency Assignment Problem (FAP)\footnote{Also known as Automatic Frequency Planning (AFP) or Channel Assignment Problem (CAP)\cite{ACOvsEA}} is a generalisation of the graph-colouring problem and is subsequently an NP-hard problem. This is because one has a finite amount of frequencies which needs to be assigned to antennae/transceivers (TRX's)  where the amount of transceivers to be assigned frequencies greatly out weigh the amount of available frequencies.

It is inevitable that a network will have interference and we can thus only minimise the amount of interference that might occur on the network - an optimisation problem. Using exact algorithms to find a solution is not practical since the time to find a solution will be polynomial. Generally Metaheuristic algorithms are used to find optimal solutions to NP-hard problems\cite{ACOvsEA}. We will discuss Metaheuristic algorithms which are generally used to find solutions for NP-hard problems in Chapter 4. 

A contributing factor to the difficulty of the FAP is due to the scarcity of usable frequencies in the radio spectrum, which forces network operators to reuse their allocated/licensed frequencies in their respective networks. The scarcity of the usable frequencies in the spectrum can be attributed to the overuse of certain bands as well as large scale reuse of frequencies in networks. This has put strain on the spectrum and has complicated the management of networks significantly because interference is more likely to occur.

Frequency assignment is the last step in a long process of network setup. Before frequencies are assigned base stations need to be placed and need to be configured, which include azimuth and tilt of the antenna for optimal network coverage. After the base stations are configured they need to be allocated a certain amount of transmitters to achieve a target network capacity \cite{AndreasPaper}.

Frequency assignment is only a means to achieve the targeted network coverage as well as network capacity.

In this section we gave a brief introduction as to what the Frequency Assignment Problem is and why it occurs as well as why it is a problem. In the next section we will describe the types of Frequency Assignment in use today and we will also discuss some of the varaints of FAP and also formally define which of the variants we will concentrate on.
\section{Frequency Assignment Types}
\label{sec:FreqAssignmentTypes}
In this section we will discuss the different methods that exist to allocate frequencies to cells in a cellular network. We will also state which method we will use through out this paper.

Within the FAP domain there exists different types of the FAP which have emerged over the years as the domain and requirements have changed. We will discuss these FAP variants in section \ref{sec:FAPIndustry}.

There are a variety of FAP in the domain (which will be discussed in later sections of this chapter) but most of these problems can be classified into two categories based on the assignment scheme they use:
\begin{enumerate}[\bf{(}a\bf{)}]
\item \emph{Fixed Frequency/Channel Assignment} (FFA/FCA) is the process of permanently assigning frequencies to cells (cellular towers). The frequencies assigned are fixed and cannot be changed on the fly while the network is active , since the frequencies assigned to the cell form part of a delicate frequency plan designed to keep interference to minimum.
\item \emph{Dynamic Frequency/Channel Assignment} (DFA/DCA) is the process of allocating channels to cells as they require it to meet the current traffic demand imposed on them by clients. 
\end{enumerate}
Each cell can be assigned multiple frequencies based on the amount of transmitters or TRX's it has. The amount of TRX's in a cell depends on the expected amount of traffic the particular cell must 
handle.

Most of the research in the FAP has concentrated on the FFA. The reason for this is because FFA is a static technique, which allows it to come up with a better solution since it has more time for calculation. FFA is also easier to implement in practise and allows the network operators to cater for the worst case scenario - heavy traffic load on the network. 

The DFA is at the moment a very hard problem because the network frequency plan is constantly changing, which means as the traffic on the network increases the longer the DFA focused algorithm will take to allocate a frequency. This increase in processing time is because the algorithm has to take into account more constraints with a lower available frequency pool. DFA must do this process within seconds since a cell needs to serve clients. 

Most researchers have concentrated on solving the FFA using heuristic approaches like neural networks, local search techniques and more recently meta heuristic approaches which include genetic algorithms, simulated annealing , ant colony optimisation and particle swarm optimisation.

In this section we have given a description of the Frequency Assignment Problem and introduced some concepts which we will use throughout the dissertation. In the next section we will present the 
Mathematics that govern the Frequency Assignment Problem.
\section{Interference}
\label{sec:Interference}
In this section our disuccsion will focus on Interference. We will describe what interference is and why it is important for cellular networks as well as give an overview of when interference occurs.

Interference occurs when frequencies assigned to connections differ by a small margin. The amount of inference on a connection defines the \emph{Quality of Service} (QoS). One can naturally make the deduction that the more frequencies differ used on connections in a area, the better quality of service one will experience in that area. 

Cellular networks use the amount of interference on their networks as qualitative measure for their QoS. A network with high interference would experience a lot of dropped connections/calls, which occurs when the interference is too high to sustain a connection or call for communication, consequently their QoS degrades as interfernece increases.

In the literature a variety of methods are given to calculate the amount of interference in a network. These methods range from theoretical approaches to precise measurements. Regardless of what method is used the end result which they all produce is called an \emph{Interference Matrix}\cite{ACOvsEA}.

An Interference Matrix concists of a number of cell pairs (\emph{i,j}), where \emph{i} is the cell receiving interference and \emph{j} the cell whose allocated channel is providing the interence. Each cell pair in the matrix has two corresponding values that indicate the level of interference if the \emph{Electromagnetic constraints} are violated \cite{Eisenblatter,Karen2004,ACOvsEA}. These values are usually normalized to be between 0.0 and 1.0 \cite{AndreasPaper}.

Primarily Interference occurs when the Electromagnetic constraints are violated, which are defined as:
\begin{description}
\item[Co-Channel] --- When a cell \emph{i} and a cell \emph{j} operate on the same frequency or channel interference will occur \cite{Eisenblatter,EfficientEvoChannelManagement,Karen2004,ACOvsEA,InterferenceOrientatedFAP}. This is called co-channel interference. This constraint is the most important constraint that must not be violated to ensure proper performance and reliability of a modren cellular network\cite{EfficientEvoChannelManagement}.
\item[Adjacent Channel] --- When a cell \emph{i} and a cell \emph{j} operate on adjacent channels, their allocated frequencies differ by one i.e. cell \emph{i} operates on channel \emph{f} then if cell \emph{j} operates on either channel \emph{f - 1} or \emph{f + 1} then adjacent channel interference will occur \cite{Eisenblatter,EfficientEvoChannelManagement,Karen2004,ACOvsEA,InterferenceOrientatedFAP}
\item[Co-Site] --- If cell \emph{i} and cell {j} are located at the same site, then their allocated frequency ranges must differ by a certain distance in the frequency domain. This distance is known as the reuse distance \cite{FixedFAPPSO,EgyptFAPPSO}.
\end{description}
A fourth constraint, kown as the Handover constraint, is also applicable in Cellular networks. This constraint imposes a separation in frequencies when one cell hands over a call to another cell. If this constrain is violated a mobile subscriber will experience a dropped call since the handover between cells fials.The above constraints only account for factors that are in our control.

Interference also occurs due to techincal limitations, natural phenonema and other external factors like other systems. Thus another constraint is imposed on the frequency that is allocated to cell. This constraint is known as the \emph{separation} constraint which imposes a minimum separation between frequencies assigned to a cell \cite{Eisenblatter,InterferenceOrientatedFAP}. To avoid clashes with other operator frequencies each cell may also have a set of locally forbidden frequencies which are not allowed to be used under any circumstance.

In this section we described what interfenece is and what the consiquences are of too much interfence in a network. We also laid out under which circumstances interference can occur in a wireless network. In the next section we will give a brief overview of in which industries the Frequency Assignment Problem is applicable.
\section{FAP in the industry}
\label{sec:FAPIndustry}
In this section we will list some of the industries where the FAP is encountered. We provide a brief overview how the problem differs compared to other industries. We will also give some references to literature where the FAP and the particular domain are discussed. 

\subsection{Satelite communication}
The FAP in the Satelite communication domain occurs with respect to the ground terminals that transmit and receive signals via a satelite. One would assume that the problem includes the satelite, but the problem is only concerned with the frequencies that the ground terminals use. It is interesting to note that the ground terminals can be a base station or a handheld device (e.g GPS or Satelite phone).

In Satelite communication, a signal is transmitted to one or more satelites via an uplink from a ground terminal. The signal is received by the recipient statelites and relayed to the interested ground terminals who receive the signals via a downlink.

The frequencies used by the ground terminals for uplink and downlink communication are separated by a large distance in the frequency domain - the typical distance is much larger than the bandwidth. When frequencies are assigned to transmitters, downlink transmitters are ignored and only uplink transmitters are considered \cite{Karen2004}. 

A radical difference with regard to the use of frequencies compared to the standard FAP in Cellular Networks is that, frequencies are only allowed to be used once. This is sepecific to the satelite domain to avoid interference\cite{Karen2004}.

There isn't a lot of research on the FAP in Satelite Communication. One of the few recent papers on FAP in this domain is a paper by Lui et al.\cite{MISatelite} where the authors employ a Chaotic Neural Network to minimize the interference in a satelite communication system with which they achieve very good results which in most benchmarks finds the global optimum.

Another papers conetrating on this domain is a paper by Houssin et al.\cite{SDMASatelite} where the allocation of frquencies assigned in the satelite system is optimised using Space Division Multiple Access (SDMA). SDMA was developed for use in 3G networks and forms part of the Multiple Access family of techniques (CDMA,TDMA) that are in use in wireless networks today. 

It is interesting to note that the authors concentrate on optimising the amount of users served by the system and not interference incurred by the allocated frequencies.

\subsection{Wireless mesh networks and WLANs}
Wireless mesh networks and WLANs\footnote{Both applications use the same standard and encounter similar problems in their respective domains.} are the most recent applications where the FAP is encountered. 

Multiple WLANs are increasingly being used to provide backbone support for large fixed line netwroks, enterprise networks, campuses and metropolitan areas. To be able to provide backbone support for these networks, a primary design goal when desiging and deploying these networks is capacity. A limiting factor for WLAN capacity is interference which affects multihop hop settings. Thus the overall network interference needs to be minimized to increase the capacity of the network \cite{MultiradioMeshNetworks}. 

Most wireless networks operate on the IEEE 802.11 a/b/g standard. A IEEE 802.11n standard is available but hasn't been finalized yet, even though one can already find wireless hardware operating on this standard. According to the IEEE 802.11g standard only 13 Frequencies are available for use and in some geographical areas a futher limiting constraint is imposed which only allows a certain subset of frequencies to be used in the particular area \cite{Karen2004}.

Typical approaches allocating frequencies include using DCA and FCA\footnote{Discussed in section \ref{sec:FreqAssignmentTypes} on page \pageref{sec:FreqAssignmentTypes}}. DCA isn't very popular since the dynamic switching of channels lowers the response time on commidity hardware since there is a delay in miliseconds when switching channels. Typical packet transmission times are in microseconds. To garuntee uptime and high responsiveness, FCA is the preferred approach \cite{MultiradioMeshNetworks}.

The FAP in Wireless Mesh Netowkrs and WLANs differ to the standard problem in that it introduces an extra constriant. Channels assigned to links on a node cannot be more than the available interfaces on that particular node. This constraint is known as the \emph{interface constraint} \cite{MultiradioMeshNetworks}. Another aspect to consider is the placement of access points (AP) in the network, which is similar to the problem cellular networks face with regard to base station placement \cite{Karen2004}.

There is a wealth of literature on Wireless Mesh Networks. In research done by Subramanian et al.\cite{MultiradioMeshNetworks} the authors formulate a lower bound using semi-definite techniques and linear programming. Using these lower bounds with their discussed algorithms they get very promising results on a simulation benchmark (ns2). Their discussed solution imposes no specific hardware or topology changes in the wireless network.

In research conducted by Chen et al.\cite{SiteFAPWLAN}, the authors follow a different route than traditional proposed algorithms with regard to interference. The authors present algorithms that focus on the interference as perceived by the user and not the Access Point (AP). The authors also use site specific knowledge provided by Blueprints, Google Earth and Google Maps in their algorithms to predict potential path loss when allocating a frequency to a AP.
\subsection{Military field communication}
In a Military context the FAP is a very difficult problem to be solved due to its dynamic nature. During deployment connections need to be established rapidly between nodes with not gauruntee that the nodes would stay static at locations. Usually nodes are military field phones can be any transceiver device. 

Due to the nature of the problem the DCA scheme is used to allocate frequencies to nodes. The Military FAP differs due to the property that any of the nodes are mobile and can move at any moment to a new location, potentially interfering with another connection. Two frequencies need to be assigned to each connection that is established, one for each direction of communication. These allocated frequencies must also differ by a certain distance in the frequency domain to prohibit alternating directions of communication interfering.

A lot of literature can be found on Military field communication. This is due to two organizations CELAR\footnote{Centre Electronique de L'Armement} and EUCLID\footnote{European Cooperation on the Long Term Defense} making data available to various research groups and allowing them to develop algorithms for frequency assignment. 

A comprehensive study by Dupont et al.\cite{DynamicFAP} on the 36 instances of real life data obtained from CELAR. The Authors state that the CELAR data actually has 3 subproblems which occurs in seperate stages. In the first stage a Constrained Satisfaction Problem is encountered when assigning initial frequencies. The second problem occurs when new links are established and frequencies need to be assigned, this is known as the second stage. The last and final stage occurs when a new link cannot be assigned a frequency and thus a repiar is needed. For each stage the authors developed algorithms to try and optimily solve it to produce and overall optimal solution.

For the instances made available by EUCLID the study by Aardal et al.\cite{CALMA} provides results from various groups who worked on the instances provided, which is known as the CALMA project. Various optimization and approximations algorithms were implemented by the research groups and new lower bounds were also found. The authors present each algorithm implemented by the underlying research group. Results are shown for Minimum Interference problems as well as Minimum Span problems.
\subsection{Television and Radio Broadcasting}
The FAP encountered in broadcasting very much resembles the problem domain found in Cellular networks. The only notable difference is the required distance allocated frequencies must differ in the frequency domain is larger in broadcasting than in cellular networks \cite{Karen2004}.

Since the problem resembles the problem found in Cellular networks, there are few articles that specifically discuss frequency assignment in broadcasting as a main topic. Research that specifically discusses FAP in broadcasting is presented by Idoumghar and Schott \cite{RadioFAP}. The authors present a distributed hybrid genetic algorithm and a cooperative distributed tabu search algorithm. They compare these algorithms with their sequential counterparts of their algorithms and with a ANTS algorithm. The benchmark instances they used were provided by the TDF-C2R Broadcasting and Wireless research center.
\subsection{Cellular Communication}
Cellular communication\footnote{An overview of a Cellular Communication technology called GSM is presented in Chapter 2.} can be considered the main driving force with regard to research in the Frequency Assignment domain. As new standards are developed and used in 3G networks, in general a frequency assignment problem still needs to be solved since these techniques do not eliminate interference entirely, but the do make it \emph{less} likely to occur.

One such technique that is mostly used in GSM networks is called Frequency Hopping, which as the names emplies occurs when the transmitters "hops" onto different frequencies according to a predefined sequence of frequencies. The frequency can change per packet if the underlying hardware can handle it otherwise it switches per connection \cite{Karen2004,MontemanniThesis,Eisenblatter}.

The FAP in the Cellular domain is the most researched topic and is considered the default domain of the problem. As such, most of the literature concentrates on this domain and one can find a lot of research in the literature presenting viable algorithms that produce real world solutions \cite{Eisenblatter}. 

Because the problem is NP-Hard most presented algorithms are either of the metaheuristic type or more recently of the swarm intelligence type. Both of these algorithmic types are discussed in Chapter 4 and 5 respectively.

Besides optimizing algorithms, there is also a wealth of literature on upper and lower bounds for the FAP. Using lower bounds in FAP orientated algorithms can produce very favourable results as demonstrated in the paper presented by Montemanni and Smith \cite{TabuMontemanniSmith}. Using a lower bound in conjunction with their algorithm they achieved a new optimum in a variety of benchmarks, most notably in the COST 259 benchmark. 

Other papers in the literature contribute by providing different modeling techniques such as the research done by Borndörfer et al. \cite{FAPOrientationModel}. The interested reader who wants to know more about the problem domain, general modeling techniques used as well as the most common algorithms used in directed to the study by Aardall et al\cite{Karen2004}.

In this section we discussed the different industries in which the Frequency Assignment Problem is encountered and also gave a brief description on how the problems are different with regard to what constraint each problem domain imposes on the frequencies for the particular industry. 
We also gave some brief references to some literature where the FAP is discussed with regard to the particular domain. In the next section we will give a brief description of the different types of Frequency Assignment Problems as well as give a small discussion on the literature found for the individual problems.
\section{Frequency Assignment Problem types}
\label{sec:FAPVariants}
We will start of giving a brief overview on one of the first and oldest problems in the domain and we  will end of discussing the problem we will base our implementation on. In this section we will give a brief explanation on some of the various problems that exist for the FAP. We will start of giving a brief overview on one of the first and oldest problems in the domain and we  will end of discussing the problem we will base our implementation on.
\subsection{Minimum Order FAP}
The Minimum Order FAP (MO-FAP) was the first FAP that emerged in the 70's. The MO-FAP is concerned with assigning frequencies to transmitters while interference is minimized as well as minimizing the amount of different frequencies that are used. 

In MO-FAP frequency re-use is prioritised and the usage of a frequency has a certain cost associated with it. The reason for this is because when the wireless network industry started out, operators were billed according to the amount of different frequencies they used. In the beginning frequencies weren't cheap since they were sold per unit \cite{Karen2004,MontemanniThesis}. 

Over the years as the law governing the wireless spectrum changed and new technology as well standards emerged, thus MO-FAP has lost its relevancy. Companies aren't billed according to the different frequencies they use, but they purchase licenses from a regulatory body. This license usually stipulates what frequency band the network is allowed to use.

In Some instances a certain band of frequencies is put up for auction by a regulatory body, to which interested parties can bid to own the specified spectrum. Due to the shift in how frequencies are allocated to network, neither the regulatory bodies nor the network operators care about the amount of different frequencies are used. Thus MO-FAP has lost its relevancy in the modern wireless industry.
\subsection{Minimum Span FAP}
The Minimum Span FAP (MS-FAP) is a problem that is very relevant today, especially when network operators want to deploy a new network in a region. The MS-FAP is concerned with keeping the interference below a certain level during assignment as well as minimizing the span. The interference threshold used, is specifed by the network designer as the minimum allowable interference on the network.

The span is defined as an interval on the frequency domain. This interval is calculated by taking the difference of the maximum and minimum frequnecies used during assignment. With the span value, network operators are able to request certain frequency bands and know their network will be able to operate at suitable interference levels \cite{Karen2004,MontemanniThesis,MSFAP}.

The MS-FAP and MO-FAP are two very similar problems, the only difference is that MO-FAP focuses on minimizing different frequencies and MS-FAP forcuses on minimizing the interval of frequencies used during assignment \cite{Karen2004}. The Philadelphia benchmark is usually used to gauge how good the algorithm performs.
\subsection{Minimum Interference FAP}
The Minimum Interference FAP (MI-FAP) or Fixed Spectrum FAP (FS-FAP) is typically encountered after the network operator has obtained a frequency band from a regulatory body. Other problems use matrices to forbid certain frequencies from being with certain transmitter \cite{Karen2004,Eisenblatter,MontemanniThesis,MultipleBinaryFAP}. 

Unlike previous discussed problems, in MI-FAP any available frequency in the allocated band may be used even though it produces interference. The other problems are concerned with the frequencies used, even though they might be violating some constraints that incur a huge amount of interference. The interference value doesn't play a large role in their respective objective functions. In MI-FAP the objective is to minimize the total amount of interference on the network. It is important to note that this amount of interfernece might not necessarily be zero \cite{Karen2004,Eisenblatter,MontemanniThesis,MultipleBinaryFAP}.

The MI-FAP is the most encountered problem currently in cellular networks, since there are more operating networks than new networks being designed in the cellular industry today. This particular problem for form the focus for this dissertation. 

Since MI-FAP is very close to real world instance problems, authors tend to use real world instances or benchmarks that resemble real world instance to test the quality and efficiency of their algorithms \cite{Karen2004,Eisenblatter,MontemanniThesis,MultipleBinaryFAP}. We'll benchmark the quality and efficiency of our solution with the COST 259 benchmark which is discussed in section \ref{sec:FAPBenchmarks}.

In this section we laid out the different types of Frequency Assignment Problems there are in the literature. We also gave a brief discussion on some of the literature found on the individual problems. Finally we formally stated on which one of the frequency assignment problems we will be concentrating on, namely the Fixed Spectrum Minimum Interference Frequency Assignment Problem (MI-FAP). 

In the next section we will give a Mathematical definition for the Fixed Spectrum MI-FAP which will form the bases for the objective/cost function that we are going to minimize to find an optimal frequency plan.
\section{Fixed Spectrum MI-FAP Mathematical Formulation}
In this section we will give a Mathematical definition of the Frequency Assignment Problem which will form the core of what our algorithm discussed in this dissertation will optimize. We'll start of by denoting the symbols we will use and then we will give the Mathematical definition of the cost function we will minimize.

The Frequency Assignment Problem can be represented as a graph colouring problem hence it is known to be NP-Complete. Before we can formally define the Frequency Assignment Problem we first need to introduce some symbol definitions.

\begin{align}
	G &= (V,E) \label{E:setG}\\
	V &= \{v_{0},v_{1},...,v_{i}\} | i \in \mathbb{N} \label{E:setV}\\
	E &= \{v_0v_1,v_0v_2,...,v_iv_j\}|v \in V,\forall ij \in \mathbb{N},i \neq j \label{E:setE}\\
	D &= \{d_{01},d_{02},...,d_{ij}\}| \forall\{i,j\} \in E, \exists d_{ij} \in \mathbb{N}^+ \label{E:setD}\\
	P &= \{\{\bar{p_{00}},\overset{=}{p_{01}}\},\{\bar{p_{10}},\overset{=}{p_{11}}\},\ldots,\bar{p_{i0}},\overset{=}{p_{i1}}\}\}| \forall \{i,j\} \in E,\exists p_{ij} \in \mathbb{N}^+ \label{E:setP}\\
	F &= \{0,1,2,3,...,k\}| \forall k \in \mathbb{N},\forall v \in V \exists f \in F\label{E:setF}\\
	d_{ij} &< |f(i) - f(j)|, \forall ij \in \mathbb{N},i \neq j \label{E:interference}
\end{align}

Let $G$ (see~\ref{E:setG}) be a weighted undirected graph, where $V$ (see~\ref{E:setV}) is a set of vertices. Each $v \in V(G)$ represents a transmitter in the frequency assignment problem. 

$E$ (see~\ref{E:setE}) is a set of edges. An edge consists of two vertices $v_i$ and $v_j$ that are joined because there exists a constraint on the frequencies that can be assigned between the two vertices or transmitters. Each edge has two associated labels $d_{ij}$ and $p_{ij}$ \cite{FAPOrientationModel,TabuMontemanniSmith}. 

The label $d_{ij}$ that is part of the set $D$ (see~\ref{E:setD}) denotes the maximum separation that is requried to exists between frequencies assigned to two transmitters $v_i$ and $v_j$. Using $f(i)$ to denote the frequency assigned to $i$, we can determine using equation \ref{E:interference} if the interference involving the transmitters $v_i$ and $v_j$ is acceptable\cite{FAPOrientationModel,TabuMontemanniSmith}

The other label, $p_{ij}$, forms part of the set $P$ (see~\ref{E:setP}) which is refered to as the Interference Matrix\footnote{Discussed in section \ref{sec:Interference} page \pageref{sec:Interference}}. Each label $p_{ij}$ contains two values which represents interference\footnote{Interference values can be zero in some cases}:
\begin{itemize}
\item $\bar{p_{i0}}$ represents the value for co-channel interference \cite{FAPOrientationModel,TabuMontemanniSmith}. 
\item $\overset{=}{p_{i1}}$ represents the value for adjacent channel interference\cite{FAPOrientationModel,TabuMontemanniSmith}.
\end{itemize}

Lastly we have the set $F$ (see~\ref{E:setF}) that denotes a set of consecutive frequencies for every transmitter in $V$\cite{FAPOrientationModel,TabuMontemanniSmith}.

Formally the Fixed Spectrum Frequency Assignment Problem (FS-FAP) can now be defined as a 5-tuple \(FS-FAP = \{V,E,D,P,F\}\) with a required mapping of \(f: V \rightarrow F\)\cite{TabuMontemanniSmith}. The objective of the FS-FAP is to find an assignment of frequencies to transmitters that minimizes the sum of total interference (see ~\ref{E:costFunction}).

\begin{align} 
 c(p_i) &= 
 \begin{cases}
	\bar{p_{i0}} &,\text{if $|f(i) - f(j)| = 0$}\\
	\overset{=}{p_{i1}} &, \text{if $|f(i) - f(j)| \leqslant d_{ij}$}\\
	0 &,\text{if $|f(i) - f(j)| > d_{ij}$}
 \end{cases}\\
 \label{E:costFunction}
 Total Interference &= \sum^\mathbb{P}_{i = 0}c(p_i),p \in P 
\end{align}

In this section we Mathematically defined the Frequency Assignment Problem using the symbols we defined. In the next section we will give a brief discussion on the different Frequency Assignment Benchmark Problems that exist and also define the benchmark we will be using in our implemention.
\section{FAP Benchmarks}
\label{sec:FAPBenchmarks}
In this sections will discuss some of the most used benchmarks in the FAP domain. We will start of with the first benchmark that was introduced in the 70s and end of with a disuccsion on the benchmark we will be using to test our implementation.
\subsection{Philedelphia Benchmark}
The Philedelphia benchmarks are derived from an instance that was introduced in 1973 by Anderson. Each instance is a hexogonal grid of cells that overlaps the area of interest. At the center of each cellthere is a transmitter. Past approaches used these hexagonal systems to model modern cellular networks \cite{Karen2004,ExactMIFAP}.

In this benchmark interference is measured by a co-channel reuse distance. This distance stipulates that the difference of the frequencies  assigned to two cells must greater of equal to a certain value $d$. A frequency cannot be assigned to a cell if it violates this minimum distance \cite{Karen2004,ExactMIFAP}.

These benchmarks are typically used to test algorithms developed for MS-FAP, since there is no concept of cost or penalty for interference incurred by violating constraints.
\subsection{CELAR}
In 1994 EUCLID introduced a project called CALMA which was a combined effort by various Europena goverments that were part of EUCLID to investigate algorithms for Military applications. The project was granted to six research groups. Within the project 36 instances were made available by CELAR for Radio Link Frequency Assignment \cite{Karen2004,DynamicFAP}.

All the CELAR instances have the constraint that the difference between frequencies assigned to interfering radio links must be greater that a certain predefined distance in the frequency domain. This is a soft constraint and may be violated. Another constraint in the CELAR instances is that each pair of parallel links must differ by an exact predefined distance. This constraint is a hard constraint an may not be violated \cite{DynamicFAP}.

These instances were initially not available to the general public as it was contained to be within the CALMA project. In 2001 the CELAR launched an the International ROADEF challenge, were certain instances from the CALMA project were made available for the research teams taking part in the challenge. The instances made available had been modified to take polarizations and controlled relaxations of certain EMC constraints \cite{LowerPolarFAP}.
\subsection{COST 256}
The COST (COoperation européene dans le domaine de la recherche Scientifique et Technique) 259 is a set of real world GSM instances made available by die European Union. The instances are publicly available and can  be downloaded for free at http://fap.zib.de/ (FAP Web 2007). The website also constains the most recent results obtained by researchers using these instances\cite{Karen2004,Eisenblatter}.

The instances are fairly difficult due to the large amount of transmitters (900 - 4000) that need to be assigned frequencies, with a relatively small amount spectrum of frequencies. The main important characteristic of this benchmark is that is resembles real world GSM network data, which is why we the authors have selected this as the primary benchmark we will be concentrating on \cite{Karen2004,FAPWeb}.

More specifically we will concentrate on a small subset of the instances that are available, namely Siemens1, Siemens2, Siemens3 and Siemens4. In the paper by Montemanni and Smith \cite{TabuMontemanniSmith} the same subset of problems is used and to date their algorithm has produced some of the best results. %Check if this is the right paper
\section{Summary}
