\chapter{Metaheuristics Algorithms}

\section{Introduction}
Metaheuristics is a sub domain of the artificial intelligence domain. It evoled out of a need for more efficient search techniques with regard to hard problems. 

Metaheuristics forms part of a collective body of algorithms that use heuristics to search a particular domain's problem space, for the most optimal solution adhering to certain hard and soft constraints. Some of the msot important Algorithms that form part of this collectivee body is:
\begin{itemize}
\item Tabu Search
\item Simulated Annealing
\item Genetic Algorithm
\end{itemize}
The above mentioned algorithms aren't the only algorithms to form part of this sub-domain, but they are the algorithms that have recieved the most attention in the literature and generally produce good results.

In this chapter our main focus will be to discuss each of the above listed algorithms. We will start of by briefly discussing the characteristics of metaheuristic algorithms after which we will discuss each of the above algorithms in detail. We will also provide a literature study for each algorithm inorder for us to see how an algorithm needs to be changed and optimised for particular problem domain. 

\section{Characteristics of Metaheuristics}
NP-Complete problems have been proven to not be solveble in polinomial by local search methods such as 1,2,3. Metaheuristic algorihtms on the other hand are much more efficient in search the problem space and produce much better results in a short amount of time.

Metaheuristic Algorihtms do not search a problem space statically by testing every possible permutation in the problem space and evaluating it. Instead these algorithms make use of local and global search methods that explore the search space probilistically; moving in the problem space to a region with a high probability of finding high qaulity candidate solutions.
\section{Tabu Search}

\subsection{Basic Algorithm}

\subsection{Literature study}

\section{Simulated Annealing}

\subsection{Basic Algorithm}

\subsection{Literature study}

\section{Genetic Algorithm}

\subsection{Basic Algorithm}

\subsection{Literature study}
\section {Summary}
